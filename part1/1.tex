四月份清朗个一日,時鐘懕
\pagenote[懕]{音ngaam1。方纔。又妥當。《說文》:「安也。」《廣韻》:「一鹽切。」《朱註》:「厭厭,安也,亦久也。」《小戎傳》:「厭厭,安靜也。」《康熙字典·心部·十四》:「又通作厭。」《說文解字注》:「釋文及魏都賦注引韓詩愔愔,和悅之皃。」有非疑母字有j-聲母與ng-聲母雙粵音,多屬影、喻、日母\parencite{chanbf97}。}%
敲響緊十三點,爲著
\pagenote[著]{音zo2。助詞,表示動作已完成。脫落韻尾之喉塞音而得今音\parencite{chanbf97}。}%
擋刺骨个風吹,溫斯頓 · smith將下爬貼住心口行,快快脆𤗈
\pagenote[𤗈]{音sip3。滑進某處。又插物入罅。}%
入勝利大廈,快毋
\pagenote[毋]{音m4。不。}%
過避免一陣塵跟彌
\pagenote[彌]{音maai4。助詞,表示同樣、補足、完畢。支韻字有讀aai,例如「徙」「舐」。}%
佢
\pagenote[佢]{音keoi5。代詞,指第三人。又作「渠」「\sbw{𠍲}」。}%
入去。

走廊有陣
\pagenote[陣]{音zan6。「陣味」詞中受「味」字影響,同化而轉韻尾,卒成為氣味之量詞。}%
煲椰菜同
\pagenote[同]{與。}%
舊地薦个
\pagenote[个]{音ge3。助詞,表示形容詞性、修飾、領屬、肯定語氣。亦代替所指示名詞。相當於國語「的」。}%
味。一頭埲
\pagenote[埲]{音bung6。量詞。}%
牆釘著張海報,喺
\pagenote[喺]{音hai2。於,在。}%
室內掛大得懠
\pagenote[得懠]{音dak1 zai6。過份。}%
,工子
\pagenote[工子]{音gung1 zai2。畫。又娃娃。}%
得隻巨大个樣
\pagenote[樣]{音joeng2。面。又外貌。}%
,一米闊有餘:一個大概四十五歲个男人塊面,鬍鬚又黑又粗,容貌健壯英俊。溫斯頓向樓梯行,費事試下搭部𨋢
\pagenote[𨋢]{音lip1。昇降機。}%
,橫㼭
\pagenote[㼭]{音dim6。豎。橫㼭,橫豎,反正。《廣韻》:「徒念切。支也。」《集韻》:「榰也。」榰,《爾雅・釋言》:「柱也。」又正。橫為惡,則㼭反為善。}%
平時都用毋逳
\pagenote[逳]{音juk1。動。}%
,而且今陣
\pagenote[今陣]{音gaa1 zan2。最近。gaa1卽「今下」之合音\parencite{chanbf97}。}%
日頭無電力,準備仇恨禮拜个節儉運動个一部分。因為到佢間單位要上七條樓梯先,成三十九歲、右腳眼上邊生靜脈曲張性潰瘍个溫斯頓,要停低幾次敨
\pagenote[㪗]{音tau2。休息。又呼吸。}%
下
\pagenote[下]{音haa5。助詞,表示短時間。}%
氣。喺個個樓梯口,對住𨋢門,海報个大隻樣注視住
\pagenote[住]{助詞,表示持續。}%
。幅畫特登
\pagenote[特登]{故意。}%
畫得好得意:人行一步,對眼好似會跟住人行。下邊標題寫住:\emph{大哥睇
\pagenote[睇]{音tai2。視。}%
實
\pagenote[實]{助詞,表示密切。}%
你}。

單位裏邊,一把甜美个聲喺道
\pagenote[喺道]{音hai2 dou6。正在。又於此處。}%
念出一連串點
\pagenote[點]{如何。從「何等物」省「何」得「等物」而合音。「等」古屬蒸韻。}%
樣關
\pagenote[關]{關於。}%
生鐵个數目。把聲從一塊長方形、似塊蒙
\pagenote[蒙]{音mung4。模糊,不光亮。《書·洪範傳》:「蒙,隂闇也。」}%
个鏡甘
\pagenote[甘]{音gam2。如此。又連詞。又助詞,修飾以後之語。}%
个鐵牌播出來,安著喺右手邊埲牆面。溫斯頓扭一粒紐,把聲細少少,不過尐
\pagenote[尐]{音di1。些。又助詞,表示某物比較如以前之語。}%
字重分得出。部機器(所謂電幕),可以校
\pagenote[校]{音gaau3。調整。}%
暗,但無
\pagenote[無]{音mou5。不存在。}%
辦法熄澌
\pagenote[澌]{音saai3。助詞,表示盡、全然。}%
佢。溫斯頓去窗前邊望下自己:瘦蜢蜢、著住其黨件藍色蛤乸衣
\pagenote[蛤乸衣]{音gap3 naa2 ji1。工作服。蛤乸,蟾蜍。}%
,令佢睇起上來
\pagenote[來]{音lai4。\textcite{chanbf97}有提。}%
重瘦。頭髮十分淺,面色天生紅潤个,皮被粗个番鹼
\pagenote[番鹼]{音faan1 gaan2。肥皂。}%
、屈
\pagenote[屈]{音gwat6。鈍。又禿。《集韻》:「渠勿切。」《說文解字注》:「鈍筆曰掘筆。短頭船曰撅頭。皆字之假借也。」\textcite{chanbf97}提過。本作\sbw{𡲬}。\sbw{𡲬},《廣韻》:「衢物切,音掘。短尾鳥也。」}%
个刀片、上年冬天个寒風癩
\pagenote[癩]{音laa2。刺激。《集韻》:「傷也。疥也。」疥,《說文》:「搔也。」}%
到𩰶歰歰
\pagenote[𩰶歰歰]{音haai4 saap3 saap3。\sbw{𩰶}、歰,不滑。\sbw{𩰶},《集韻》:「何開切,音孩。」《類篇》:「麧也。一曰糜中塊。」比較:糙,《廣韻》:「米糓雜。」}%
。

由道閂%
\pagenote[閂]{音saan1。關閉。}%
著个窗望出去,出邊世界好似好冷淡。下邊街道,風喺道卷起尐塵同尐爛紙。日頭雖然曬住、天又藍得𥋇眼,但物
\pagenote[物]{音mat1。何。\textcite{chanbf97}提過。}%
也
\pagenote[也]{音je5。物。}%
都好似無物色甘,周街尐海報之外。箇
\pagenote[箇]{音go2。遠指示詞。彼。}%
隻黑鬚僗
\pagenote[僗]{音lou2。傢伙。}%
由個個居高臨下个角度望落來,連對面間屋都有一張。標題寫住:\emph{大哥睇實你},箇對眼深望住溫斯頓自己箇對。下邊街頭,第張海報,一個角扒
\pagenote[扒]{音mit1。撕。《集韻》:「筆別切,\sbw{𠀤}讀若分別之別。擘也,剖分也。」}%
開著,被
\pagenote[被]{音bei2。}%
風吹到擺來擺去,斷續露出兩個字:\emph{英社}。好遠下,有部直昇機喺道穿過尐屋頂之間,好似隻烏蠅甘懸停一陣,然後側住一邊飛走。巡邏來个,道道窗都𥊙
\pagenote[𥊙]{音zong1。偷視。}%
入去。但其實尐巡邏都毋使驚。尐思警先至
\pagenote[先至]{僅僅。又省作先,又省作至。}%
驚。

溫斯頓背後,部電幕繼續儑
\pagenote[儑]{音ngap1。胡說。}%
三儑四關生鐵同彌第九個五年計劃个超額完成。電幕同時傳播同錄音个。所有溫斯頓整个聲會收到,除著極之低沉个耳語,而且,喺塊鐵牌个視野之內,亦會被人見到。當然,無辦法知邊個幾時喺道睇住你。無人知思警會幾常時或順住物也系統揀邊個來睇。有可能所有人幾時都一切望住澌添。總之,佢敵
\pagenote[敵]{音dei6。《說文》:「仇也。」仇,先指對方,後指敵人。《廣韻》:「匹也,當也,輩也,主也。」脫落韻尾得今音。}%
隨時都可以揀你。係一個習慣——何止習慣,本能感——當任何聲都會被人聽到、當任何動作,除非黑暗箇時,都會被人檢查个生活。

溫斯頓背住部電幕,寧願安全尐;而佢夠知背脊都識得透風个。一公里之外,白色个高聳个眞實部,溫斯頓个工作場所,從邋邋遢遢
\pagenote[邋遢]{音laat6 taat3。不整潔。}%
个風景凸出來。含有模糊个討厭,溫斯頓喺道恁
\pagenote[恁]{音nam2。思念。\textcite{chanbf97}提過。}%
:倫敦,戰地機場一號个最主要城市。戰地機場一號經已係大洋洲个第三最高人口个省。溫斯頓盡力試下記起細個箇陣尐回憶,確定倫敦係毋係不留都甘。係毋係不留都有一棟又一棟腐爛十九世紀式个屋、尐牆要用木板頂住、尐窗用紙板同彌尐屋頂用波紋鐵補住、尐圍牆𦖿
\pagenote[𦖿]{音dap1。垂下。\textcite{chanbf97}提過。}%
𦖿地个?重有尐炸過來个廢墟,尐灰塵飄下飄下,尐柳葉菜喺尐瓦渣上邊周圍生;又有尐被炸彈炸平著个大尐个地方,起著幾羣寒酸个成個雞籠樣个木个棚屋,箇尐都係毋係不留喺个?無用,佢記毋到:佢尐童年經已毋記得澌,除著一連串光芒芒个活人畫,後邊無背景,通常毋清毋楚。

眞實部,現代話(\emph{大洋洲个官方語言。語法同語源,請見結論。})叫眞子,同任何見到个也毋同得好交關。一座巨大白雪雪个石屎金字塔,旁邊有無數个騎樓,飆
\pagenote[飆]{音biu1。昇。}%
上天空三百米高。由溫斯頓个位置,懨懨睇得出刻著喺面,其黨三句口號:
\begin{quote}\emph{
戰爭係和平\\
自由係奴役\\
無知係力量
}\end{quote}%
聽講,眞實部地上有三千間房,地底有相當个??ramifications??。分散喺倫敦,重有三座相似而同樣大个建築物。喺佢敵下邊,隔籬尐樓變著好細粒,甚至由勝利大廈屋頂一望就望得到全四個。佢敵就係構成成
\pagenote[成]{音seng4。全部。}%
個政治機構个四個部門尐大樓。眞實部,關注新聞、娛樂、教育、美術。和平部,關注戰爭。愛情部,保持法律秩序。大把部,負責經濟。現代話尐稱呼:眞子、和子、情子、大子。

愛情部先恐怖,一道窗都無。愛情部,或半公里以外,溫斯頓從來未入過。係一個有本事先入得去己地方,而甘都要穿過一個滿鐵絲網、鐵門、隱蔽機場陣地个迷宮。連縛
\pagenote[縛]{音bok3。連接。}%
住外牆个幾條街都有黑衫黑面个警衛,個個都抯
\pagenote[抯]{音zaa1。持。\textcite{chanbf97}提過。}%
支椎去巡查。

溫斯頓突然擰轉面。佢个表情經已變著冷靜个樂觀,最適合對住部電幕。佢穿過間房入細細間个廚房。甘早返
\pagenote[返]{音faan1。回歸。又助詞,表示恢復。}%
屋其
\pagenote[屋其]{音uk1 kei2。家,住所。其,虛義。}%
就犧牲著飯堂套餐,而佢知廚房一尐也食都無,除著一具
\pagenote[具]{音gau6。量詞。中古虞韻字多來自上古侯韻\parencite{chanbf97}。}%
要留俾
\pagenote[俾]{音bei2。讓,許。又益。}%
天朝早
\pagenote[天朝早]{音ting1 ziu1 zou2。明日晨早。ting1卽「天光」之合音\parencite{chanbf97}。}%
食个黑麵包。佢由架上邊攞
\pagenote[攞]{音lo2。《集韻》:「揀也。」}%
一樽透明个液體,簡簡單單个白色牌印住\emph{勝利氈}。陣味油到想嘔,好似米酒甘。溫斯頓斟差毋多成個茶杯甘多,鼓起勇氣,好似食藥甘吞著佢。

卽刻塊面紅澌,猛滮
\pagenote[滮]{音biu1。水由小孔出來。}%
眼淚。口感好似飲硝酸甘,而且一吞,後尾䪴
\pagenote[後尾䪴]{音hau6 mei5 zam1。頭後部。}%
就有被人用支橡膠棍攴
\pagenote[攴]{音bok1。\textcite{chanbf97}有提。}%
過來个感覺。不過,下一刻,肚裡邊个燒灼感覺散著,世界又好似歡樂著一尐。從一個䩌
\pagenote[䩌]{音caau4。有紋路。彭志銘提過(2009)。}%
个印住\emph{勝利煙}个盒,攞著一支煙出來,毋覺意打直甘抯,俾尐煙草跌澌落地。第二支成功尐。佢返返入廳,喺電幕左邊張檯子坐低。由檯桶攞出來一個筆架、墨水同埋一本厚个空个四開本个書,後邊紅个,書皮有大理石纹彩。

毋知點解,電幕个位置好奇怪,無平常甘安喺埲??end wall??,俾佢睇到成間房,反而安著喺長个箇道,對住道窗。側邊有個淺个角落頭,溫斯頓坐緊
\pagenote[緊]{音gan2。助詞,表示繼續。}%
喺道,建立箇時應該整來擺書架个。溫斯頓一坐喺爾個角落頭裏邊,挨得好後,睇起想來,電幕就睇毋到佢了
\pagenote[了]{音laa3。於問題,變陽平聲。}%
。當然重俾佢聽得到,但係如果佢留喺而今
\pagenote[而今]{音ji4 gaa1。}%
个位置,就睇毋到。某種程度上係間房个古怪个結構令佢恁起佢將做个也。

但係啱攞出來箇本書又有令佢恁起。一本特別令
\pagenote[令]{音leng3。漂亮。}%
个書來个。奶白色个開始發黃个書頁,用或者四十年多無再製造个紙造个,而溫斯頓認爲本書應該老好多。佢喺某個貧民區(邊一區又毋記得著)个一座??frowsy little??古董鋪
\pagenote[子]{音zai2。\textcite{chanbf97}有提。}%
見到,卽刻渴望買著佢。雖然黨員毋應該入普通鋪頭(所謂「自由市場交易」),其實??rule was not strictly kept??,因爲好多樣也無第尐方法去攞,例如鞋帶同刀片。佢兩頭望著一下先,就𤗈入去,畀
\pagenote[畀]{音bei2。給,予。}%
著兩個半,買著佢。當時未覺得有物也原因去買。佢鬼鬼鼠鼠甘喺個篋
\pagenote[篋]{音gip1。\textcite{chanbf97}有提。}%
裏邊拎
\pagenote[拎]{音ling1。持。}%
著返屋其。雖然空个,但係都係一個有罪个所有物。

佢將做个也,就係開始寫日記。雖然毋犯法(無犯法也,因爲經已無法律了),但係假如被人敵知,肯定會被人判處死刑,毋係就起碼罰廿五年勞改。溫斯頓安個筆嘴喺筆架道,啜住佢來整𠞉
\pagenote[𠞉]{音lat1。脫落。}%
尐油面。筆係古老个器具,簽名都少見。溫斯頓偷偷地、有少少困難感,攞得到,剩因爲佢覺得甘
\pagenote[甘]{音gam3。如此,甚。}%
令个奶白色个紙應該用支眞筆來寫,靡
\pagenote[靡]{音mai5。莫,勿。}%
用個普遍个墨筆。其實佢毋太習慣親手寫也。除著好短个便條,通常物也都對住部話寫講,當然爾個情況毋得。佢沾
\pagenote[沾]{音dim2。將物投入液體。}%
支筆喺樽墨裏邊,就猶豫著一秒鐘。??a tremor had gone through his bowels??。一點張紙就??decisive act??。好細、好馬虎甘寫:
\begin{quote}\emph{
一九八四年四月四日
}\end{quote}%
佢挨返後。佢突然間覺得完全無助。首先,佢一尐都毋肯定今年係一九八四年。佢知個日子差毋多,因爲佢都幾知佢三十九歲,同彌相信佢一九四四或四五年出世个;而今陣,日子無可能準確過一兩年以內。

忽然間恁起,究竟,爾本日記係寫俾邊個个呢?俾未來,俾未出生个。佢考慮一陣紙上邊个可疑个日子,就撞到箇個現代話字「雙恁」。佢終於明白所做个也个嚴重性。究竟點樣同未來溝通呢?自然無可能。抑係未來會似而今,甘毋會聽佢話;抑係毋會似而今,甘佢而今个困境會對佢敵無意思。

獃著喺道幾耐
\pagenote[耐]{音noi6。久。《前漢·高帝紀》:「功臣侯表,宣曲侯通,耏爲鬼薪,則應氏之說斯爲長矣。」鬼薪,秦漢時勞動徒刑。}
下,望住張紙。部電幕轉著播刺耳个軍歌。佢覺得好怪,毋單止似乎失著表達自己个能力,甚至想寫物也都毋記得彌。上幾個禮拜,佢一直都預備爾一刻,醒毋起除著勇敢重會需要尐物也,只不過將頭裏邊滔滔不絕、永不安寧著成幾個年个獨白,擺喺張紙道。但係爾一刻連個獨白都停彌。兼且,佢个潰瘍開始痕得好緊要。佢毋敢擾
\pagenote[擾]{音ngaau1。搔,撓。上述j-聲母與ng-聲母雙粵音理論。口語變陰平聲。}%
,因爲佢一擾就腫起。一秒又過一秒。佢剩留意前邊張白紙、腳眼上邊个痕、巴巴閉个音樂、尐氈引起个小小醉感。

突然間佢慌失失博命寫,毋係太明自己寫緊尐物也。佢个又細又似細僗子
\pagenote[細僗子]{音sai3 lou6 zai2。兒童。}%
个字跡十分之潦草,一路寫一路逐漸退化,以至句號都無彌:
\begin{quote}\emph{%
一九八四年四月四日。昨晚
\pagenote[昨晚]{音cam4 maan5。cam4卽「昨暝」之合音\parencite{chanbf97}。}%
去著睇戲。全部係戰爭片。有一套好正个講一個載滿難民个船喺地中海邊道被人炸到。觀眾覺得好搞笑見到個肥僗游緊水被直昇機追住,先見佢成隻海豚甘喺水道轆
\pagenote[轆]{滚動。}%
,然後經過直昇機个瞄準線望佢,甘射到佢穿澌竉尐水染澌紅色就突然間沈著好似尐竉入著水甘,沈箇時觀眾哄堂大笑。甘就見到個救生艇裝滿細僗子,有個中年女人(可能係猶太人)坐喺船頭抱住大約三歲个細僗。細僗驚到喊挃個頭喺女人尐胸之間似乎想蜎
\pagenote[蜎]{音gyun1。爬。《分韻撮要》:「蟲行貌。」}%
入佢甘女人攬實佢安慰佢雖然佢自己驚到面都靑微一路盡可能遮住個細僗似乎以爲擋得到尐指彈。直昇機抌個廿公斤炸彈落去佢敵道閃十分光芒艇得廢柴。甘影得好精彩一個細僗子隻手飛飛飛上天肯定前邊有錄影機个直昇機跟著佢上黨員尐位道就好大掌聲但係喺戲院个無產箇邊有女人無端端發爛謯
\pagenote[發讕謯]{音faat3 laan6 zaa2。抵賴。讕,《說文》:「怟讕也。」謯,《廣韻》:「謯訝訶皃。」}%
話佢地毋應該喺細僗子面前播佢地毋毋道德毋好喺細僗前毋道警察鏟鏟著佢走無也卦佢話之尐無產點講尐無產先會甘無聊佢敵不留都
}\end{quote}%
溫斯頓停著,部份原因係因為佢抽筋。佢毋知物也令佢滔滔不絕甘寫甘个垃圾出來。奇怪个就係,寫緊箇時,同時記得起好淸楚一件完全毋同个事,淸楚到就來想寫低。佢意識到,原來因爲爾個事件,佢今日突然間返來屋其開始本日記。

如果甘朦朧个事情眞係有發生个話,就喺箇日部門朝早發生。

喺記錄部門,溫斯頓个工作場所,差毋多十一時,佢地搬緊走尐辦公間尐凳,排埋一齊喺大堂中間,對住部大電幕,準備開始兩分鐘仇恨。溫斯頓啱啱喺中間某一排坐緊低,突然間認得兩個人入來,從來未同佢地傾過計。第一個係常時喺走廊旁邊經過个女子。雖然毋識佢个名,但係知佢喺小說部門做也个。既然有時見到佢尐手油𦛚𦛚
\pagenote[油𦛚𦛚]{音jau4 nam6 nam6。油膩。}
个、抯住支士巴拿,或者佢有份關尐寫小說个機器个機械工。睇起想來膽大、大概廿七歲、頭髮厚、樣有雀斑、??switft, athletic movements??。一條窄个鮮紅色腰帶,靑年反性聯合會个標誌,丩
\pagenote[丩]{音kiu5。纏繞。}%
住件蛤乸衣个腰部幾次,啱啱夠緊突顯佢條勻稱个腰。溫斯頓一睇到佢就睇佢毋起。佢知點解。係佢成日保持住个氣氛,令人恁起冰棍球場同??cold water baths凍水涼??同社區遠足,總之全體个純潔。溫斯頓對差唔多所有女人都睇毋起,尤其是又令又後生箇尐。不留都係尐女人,尤其是尐後生女,最盲信其黨个、口號吞澌落肚个、成日掛住做間諜、聞出任何非正統。但係佢覺得爾個女子特別危險。有一次佢地喺走廊經過大家,個女子側個頭望佢一眼,好似睇穿著佢甘,令到佢滿心黑色个恐怖。思警員來个,都有恁過。講眞,都無物可能。但幾時喺佢附近,佢都仍然覺得一個古怪个心慌,同恐懼同敵意混合。

另一個係位男人叫奧巴仁,內黨員,職位又巴閉又遠離得溫斯頓對尐性質覺得含糊。大家一見到內黨員件黑蛤乸衣埋來就收聲。奧巴仁大隻、頸粗、面又歡樂又野蠻。雖然外貌甘得人驚,但係佢有佢自己種魅力。佢整正幅眼鏡箇時,迷人得奇怪,難以確定个方面上,斯文添,十分細微个一個小動作。箇陣重有人會甘恁个話,會覺得佢似一位遞出鼻煙盒个十八世紀紳士。十二年以內,溫斯頓可能見過佢甘多次。佢覺得佢好吸引,毋單止因爲佢个文雅个態度同強壯个體格有甘大个差異,而首先溫斯頓心裡懷疑(或者希望至啱)奧巴仁个正統毋完善。佢塊樣某道令佢??irresistibly毋停??甘恁。又或者佢个表情毋代表非正統,而??intelligence??。點都好,佢似一個同佢傾得計个人假如捃
\pagenote[捃]{音wan2。尋,求。}
到方法避開尐電幕、單獨同佢一齊。溫斯頓完全未出過力去查實%this is legit canto
:的確,無辦法去做。而今,奧巴仁就望下佢隻手錶,發現就來十一時,擺明打算留喺記錄部門,過彌兩分鐘仇恨先。佢坐喺溫斯頓同一排,隔幾個位。隔籬辦公間个一個細粒黃頭髮个女人坐喺佢敵之間。深色%dark was not mentioned before...
頭髮个女子坐喺溫斯頓後邊。

下一刻,房另一頭部電幕開始播出一場可惡、似一部無加油个機器甘逆耳个演講,一個??倒牙set one's teeth on edge??、令頸後邊尐毛豎起个??噪音noise??。仇恨開始著了。

照舊,人民公敵Emmanuel · Goldstein塊樣喺電幕个畫面出現。觀眾到處有人噓。細粒黃頭髮女人又驚又??disgust反感??到吱一聲。Goldstein係叛徒、慣犯,好耐以前(幾耐,無人好記得),屬於其黨領軍人物之一,就來有大哥甘偉大,就開始從事革命活動、被其黨判死、神神祕祕感走𠞉
\pagenote[𠞉]{音lat1。甩,脫落。}
著,失著蹤。兩分鐘仇恨个節目日日都變,但係無一個葛士田毋係最主要个人物。背叛之一、其黨最初个染污,所有以後反黨行爲、所有背叛、怠工、異端、偏離,出自由葛士田尐邪教。不知何處,佢繼續生,繼續圖謀:或者喺海外,喺佢尐外國主計官个保護之下,或者(時時一句是非)重喺大洋洲裏面。

溫斯頓個隔膜縮緊著。佢無可能見到葛士田个樣而毋覺得一個鋒利个一混感情。瘦个猶太樣、一頭白色毛毛、細細個羊咩鬚,睇上來一個好醒个樣,而毋知點解又賤格,鼻長得癡癡地,尖托住副眼鏡。似隻羊感个樣,把聲又有羊个性質。葛士田又喺道䛅佢个狠毒个攻擊對其黨个學說,誇張同變態到細僗哥都應該睇得出,而啱啱夠似眞到令自己驚人地無感淸醒,可能會信佢。佢喺道侮辱佢,喺道詰責其黨个獨裁,喺道問取同歐亞洲个和平卽刻結束,喺道提倡言論自由、新聞自由、集會自由、思想自由,佢喺道猛嗌革命出賣著,全部一流又一流感滮,整蠱其黨時常時个講法,現代話个辭彙都有,的確多過一個黨員會平常用个。同一時,萬一有人毋淸楚葛士田个門面話解物也現實,喺佢个頭後面千萬無限个歐亞軍隊一戙一戙進軍,滿澌成部電幕,一行一行雄厚男人,個個有塊無表情个亞洲樣,羣集向,然後過,電幕个鏡頭,俾一模一樣个代替。腳步个鼓點做葛士田狂叫个氣氛。

仇恨未到三十秒鐘,人一半經已喺道鬥氣大聲嗌。得戚个羊樣,同埋後面歐亞軍个威力,實在受毋住,一恁起或見到葛士田都經已會令人又惱又驚。人憎佢重多過憎歐亞定東亞,反正大洋洲同一個爾尐大國打緊仗个時候,通常都同另一個和平相處。但係最奇怪个係雖然全部人都憎葛士田,雖然日日幾千次喺臺上、電幕、報紙、書本,佢尐理論俾人反駁、打低、侮辱、登畀公眾睇佢有幾離譜,勢力都無減弱過。不留都有新个豬頭丙等緊俾佢𧨾。無一日丙去思警毋會揭穿葛士田指令个間諜同怠工者。佢指揮一個又巨大又奧妙个叛軍,一個地下起義串謀者个網絡,風聞話叫兄弟會。又有嘴逳,有一本好得人驚个書,收集所有異端,葛士田寫个,爾處箇處靜靜雞傳來傳去。無名个,人剩叫佢做,如果敢提个話,「箇本書」。但人係剩聽是非八卦至知啫。可以避免个話,普通黨員毋使䛅兄弟會或箇本書就好啦。

到第二分鐘仇恨變激烈。人喺位道跳跳紮、大聲喊叫,爲著停止箇把聲喺佢地个頭裏面咩。箇個細粒沙毛女成面紅澌,把口係道開開閂閂似條離水魚。奧巴仁个大隻樣都紅,喺張凳道坐得好直,身一路震感呼氣,猶如佢要對住波浪徛穩。後邊深色頭髮個女子開始尖叫:「你死豬!爛豬!臭豬!」,忽然間攞著重个一本現代話字典,𠌸著佢去電幕道。撞到葛士田个鼻哥就彈返落地,把聲滔滔不絕感繼續。溫斯頓突然醒起,原來佢都同佢地一齊叫,暴力感踢住凳腳。兩分鐘仇恨最惨个毋係必須參加,而係無可能毋參加,三十秒鐘裏就扮都毋使扮。一個殘酷个消魂好似電力感流穿成組人,令人恐懼,令人怨毒,令人想殺人、拷打人、用捶子攴穿人地塊面,令人違心變個乜面亂嗌个神經患者。但係個人覺得个仇恨係一個抽𧰼、無方向个感情,點蠟燭感易可以調佢去另一個主題。所以,一刻溫斯頓个仇恨一尐都毋係對葛士田,反而對大哥、其黨、思警,此時佢侹佢傷心,爾個孤獨、成日俾人笑个異教徒,眞實正經个獨一個守護者喺爾個大話世界。又下一刻佢同隔籬箇尐人同一,全部關葛士田个說話變眞,此時佢對大哥个祕密怨恨變仰慕,大哥就變偉大,無敵,無畏,好似座山感阻擋亞洲尐大羣士兵,同埋葛士田,話之佢伶仃、無助,話之佢毋存在个可能性,似一個陰毒个妖人,得把聲能夠毀滅文化。

有時,自動轉個人个仇恨向爾樣箇樣都可以。忽然間,好似發完惡夢彈起身感衝動一個動作,溫斯頓成功調佢个仇恨由電幕塊樣向後邊个深色頭髮女子,就見到精彩、美麗个迷幻。佢會用支橡警棍攴死佢。佢會綁住佢个赤身喺條柱道,好似聖巴斯弟盎感射穿佢一拃箭。佢會強姦佢,達高潮就斬開佢條頸。今次最正个,佢終於明白佢點解感憎佢。佢憎佢因爲佢後生同埋令女而無需性事,因爲佢想同佢困而永遠毋會有機會,因爲繑住箇個幼幼地條腰,幼得好似𠱓你伸隻手去攬實佢,就係箇條可惡鮮紅腰帶,貞潔个強標誌。

仇恨就達高潮。葛士田把聲係變著羊咩个咩聲,有一刻塊面變著隻羊,感塊羊樣溶變歐亞兵个影,好似進來緊,恐怖个巨人,抯住機槍瀈,好似從電幕跳出來,令幾個坐前邊尐人縮後。但係,敵人就變著大哥,黑頭髮,黑鬍鬚,滿面力量同沈靜,面大隻到就來滿澌成個鏡頭。無人聽得到大哥講物也,幾句鼓勵,喺戰中䛅个也,分毋出個個字,但係見到人講就補返個人信心。感大哥个樣就消失,其黨三句口號粗體出現:
	
戰爭係和平
自由係奴役
無知係力量

但係人人都重覺得見到大哥个樣喺部電幕道,好似留著個印𧰼喺佢地个眼珠道,過著一陣先彈返起。個細粒沙毛女人趴著喺前邊張凳个挨屛道。震住䛅條句好似「我个救主!」,伸出佢對手向電幕,然之後篢住佢塊面,顯然喺道祈禱。

爾一刻成組人開始又低沈又緩慢感吟:「哥...哥!...哥...哥!」又講又再講,慢慢,第一同第二個「哥」之間有好長个停頓,一個沈重个和諧,有少少野蠻,聽到赤𨂽腳打鼓感滯。繼續著或者三十秒鐘都有。非常感動个時候成日聽到个一首歌謠。一部份係尊敬大哥个智慧同偉大,而更加係種自我催眠,用鼓點打低自己个意識。溫斯頓覺得佢个腸胃結冰。兩分鐘仇恨箇時佢忍毋住同大家癲,但係爾場無人性感吟「哥...哥!...哥...哥!」次次都令佢恐怖。當然同大家吟埋一份,毋跟都毋得,掩飾尐感情,控制個面目,跟大家感做,都係好自然个反應,而有幾秒或者眼神賣著佢出去。就爾一刻發生,眞係有發生个話。

同奧巴仁眼光相接著一陣間。奧巴仁徛緊向道,帶返對眼鏡箇個小動作之中。但係一秒半秒佢地眼光相接,箇一刻溫斯頓就知(眞係知!)奧巴仁同佢恁得一樣。係個搞毋得錯个傳情,似乎佢地兩個个頭腦變透明,思想一個去第個穿佢地尐眼感流。「我同你同意,」奧巴仁好似同佢感講感。「我明白澌你所覺得。我明白澌你个輕視、你个仇恨、你个嫌惡。但係毋使驚,我喺你身邊!」感就箇個曉得從佢塊面消失,面情變返大家感不可測。

係感多。溫斯頓經已懷疑無發生過。爾尐節目不留都無後段,剩養佢个信念,或者希望,有第尐人好似佢一樣係其黨个仇人。或者宏大个地下串謀尐風聲係眞个,或者兄弟會眞係有!雖然不斷有拘捕、自認、正法,都無可能知兄弟會毋係剩一個故子,有時佢信,有時毋信,物證據都無,惟有眼可能睇到尐也,可能有無意思,抑係竊聽到幾個字,抑係廁所埲牆輕輕畫著一兩也,又有一次見到兩個陌生人撞到,一個細細个手勢,可能表達認識。全部齋估,應該全部喺佢想𧰼中只。佢返返佢个小房間無再望過奧巴仁,同佢跟進差毋多無恁過。恁到點做都會十分危險。一兩秒,佢地彼此望一望,就完啦。但係感都算値得記念,爾個孤獨監獄中。

溫斯頓醒返少少,坐直尐,呃一聲,尐氈昇返上來。

佢對眼又望實張紙,發現原來佢坐喺道考慮箇陣,佢又喺都好似自動感寫也,又毋係之前箇尐擠逼、醜怪个字,佢支筆一跣,跣過張字,寫著一行又一行:

	打倒大哥打倒大哥打倒大哥
	打倒大哥打倒大哥打倒大哥
	打倒大哥打倒大哥打倒大哥
	打倒大哥打倒大哥打倒大哥
	打倒大哥打倒大哥打倒大哥

再寫再寫,寫著滿半頁。

佢忍毋住覺得有尐心慌慌。都罷啦,反正寫箇尐字又無開本日記更加危險,但係有一陣間佢驚到好想撕開澌染污著箇尐頁,放棄成件事。

不過佢無感做,因爲佢知係經已無用。寫「打倒大哥」,抑係毋寫,都無分別,繼續寫,毋繼續,無分別,思警一樣會拐佢。佢犯著(物都無寫都犯著)基本、含所有其他犯罪个罪,所謂「思罪」。思罪毋係一個永遠都收得埋个也,或者可以避到幾何,幾年都可能,但係遲早都捉到你。

不留喺夜晚,不留尐拘捕夜晚發生。突然醒起身,粗暴个手搖住個膊頭,好嚴尐面圍住張牀。大多數都無審訊,無報告,人剩感消失著,次次都夜晚。名從戶籍道挽著,個個做過尐也尐記錄擦著,存在俾人否認,然後完全毋記得,俾人地革除著,消滅著,常話「揮發著」。

佢又突然驚起,又開始猛亂寫也:

佢地會射我話之佢佢地會射我頸後面話之佢打倒大哥佢不留都會射你頸後面話之佢打倒大哥

佢覺得有少少羞恥,坐返佢張凳道,放低枝筆。下一刻,佢嚇到彈起。有人敲門。

感快脆!佢坐得好似隻老鼠感靜,死都希望邊個試著一次之後就會放棄。無今好彩,門繼續敲。最衰就會係拖長佢㗎啦。心雖然打鼓感跳,但塊面,依住長久个習慣,應該都無表情。起身,慢慢向道門行。
