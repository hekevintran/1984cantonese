一九八四
佐治 ‧ 奧威爾 著
林花 譯
\document{article}
序
本文,係佐治奧威爾之著作《Nineteen Eighty-Four》翻譯變廣東話。爲推廣廣東話做個日常寫語言,又減低對寫廣東話輕視,譯者然爲,而希望讀者注意是而讀。絕對寫得出,並可以斯文!
譯者之目的,非推廣本文特別書廣東話之方式,包括何其多非常假借與古字,而推所謂「標準廣東文」之可能。譯者意見,廣東文之標準必合以下條件:
(一)見得:「口加音」字少、假借少、盡量用傳統字、新造字部首吻合
(二)寫得:選字不選複雜得太離譜(例:「篢」、「㔶」,兩個可讀/kam2/,解覆蓋,撰先個)
(三)懂得:字字之選擇易解、用與其他詞時不使讀者糊塗(反例:嘅作「忌」)
如此,本文之方式可能合此三條例。
是否本字有遠慮稱呼,因為好多音有幾個字音義適以代表。盡量用合音義己字以致講所有方言者亦睇得明,有時或者查下字典學下新字。

預防讀者以本文難讀,請照右手註釋。
本文字體與常見白話有甚大差異,強建議先覽尾部之註釋。

Or click here
好奇怪,原來感多音會有字如果精母歸端母
像

本字越用得多,多尐種人就睇得明。各種方言就可以毋躊躇而寫。


\end{document}
