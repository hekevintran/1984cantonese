附錄
現代話尐原則

現代話,係大洋洲己官方語言,爲滿足英社(或稱英式社會主義)尐意識形態上己需要而創造。一九八四年,未有人以現代話爲惟一個交流方式,??neither in speech or writing??。雖然時報尐頭版項目會用,其實專家先做得出己??tour de force??。預計至大概二零五零年(我地所謂正式英語),現代話將完全代替古代話。至箇時,佢漸漸??gained ground??。所有黨員??everyday speech??中有多用現代話己詞彙同??grammatical construction??己傾向。一九八四年己、俾第十與十一版現代話字典??embody??己??version??,係??provisional??,包好多後期將廢除己多餘詞彙同古老??formation??。於此,我地要關注第十一版字典??embody??己、最後己、完壁己??version??。

現代話己目的,係毋單止畀尐??devotee of ingsoc??一個??medium of expression for the proper (orthodox) world-view and mental habits??,而且??令其他make all other modes of thought impossible??。??it was intended??,現代話??adopted once and for all??而古代話毋記得著後,異端——卽由英社尐原則偏差——己念頭,會無可能恁到,??as far as thought is dependent on words??。詞彙設計得令所有黨員有??exact??而十分微妙己方法來表達全部正經己意思,同時??exclude??其他意味同尐??到達佢地the possibility of arriving at them (meanings) by indirect methods??,??partly??透過創造新字,而首先透過??eliminate (meanings)????undesirable??己字同??strip??尐剩餘己字尐非正統意味,而盡量淸埋所有??secondary meaning??。舉例:「free」(「自由」)字仍流通於現代話,而剩用得於「This dog is free from lice」(「爾隻狗無蝨乸」)「This field is free from weeds」(「爾個田無野草」)等句,毋用得喺「politically free」(「政治自由」)「intellectually free」(「知識自由」)等舊意義上,因為政治自由、知識自由己概念已毋存在,則必無名。縮細詞彙俾人視爲本身應做,一個捨得己字都無剩,??與壓抑顯然異端己字quite apart from...??。現代話己目標,非放大而縮細??range of thought??,受減低詞彙尐選擇己間接幫助。

現代話基於我地熟識己英語,不過數多現代話己句,今時今日識講英語己人會覺得非常難瞭,連無新造詞彙亦如此。現代話字分類有三:甲詞彙、乙詞彙(別稱複合詞)、丙詞彙。逐個分類談論會簡單尐,而成個言語尐特性於甲詞彙條解釋得澌,橫㼭同一尐規律對每分類仍然拏更。

甲詞彙:由日常生活需要己字組成——飲食、打工、著衫、上梯落梯、坐車、種花、烹飪等等,幾乎全部字已有己,譬如打、走、狗、樹、糖、屋、田——而照目前英語詞彙來講,十分少、定義嚴格好多。??All ambiguities and shades of meaning??淸徐著。此分類尐字,形容一個明確己概念己斷續己聲而已,毫無文學、政治、誓學上己用途,惟表達得單純而故意己念頭,通常關實物或??physical action肢體動作??。

現代話己語法有兩個突出特性。其一係??different parts of speech??幾乎完全互換。某字(原則上,「如果」「幾時」甚抽象己字都得)可用作動詞、名詞、形容詞或副詞。動詞與名詞,詞源相同箇時,形式毫無分別。此規律而已,消滅著數多古老形式,譬如,現代話無「thought」(「思念」)字,俾「think」(「恁」)取代,當名詞與動詞用。無順某詞源上己原則而變:有時留原本己名詞,有時留動詞。連相關意義而毋同詞源己名詞與動詞,常取一而捨其他,譬如,無「cut」(「切」)字,既其意適合俾動名詞「knife」(「刀」)包含。形容詞以加「-ful」(「己」)後綴喺動名詞尾而造,副詞可加「-wise」(「感」)。於是乎,「speedful」卽「快」,「speedwise」卽「快脆感」。一尐目前己形容詞仍有,例如「good」「strong」「big」「black」「soft」,而總共極少。幾乎全部形容詞由加「-ful」喺動名詞尾而造得,故無物需要。現存己副詞廢澌,除一尐原有「-wise」喺尾己字:「然」詞尾??invariable (always applicant)??己。譬如,「well」(「好好地」)俾「goodwise」取代。

況且,每字——再講,言語中某字都得——可以加「un-」前綴而否定,抑係加「plus-」前綴而強調,抑係加「doubleplus-」來更進一層。感,「uncold」(「毋凍」)卽「warm」(「暖」),「pluscold」「doublepluscold」卽「好凍」「極之凍」。與目前英語相似,可以「ante-」「post-」「up-」「down-」等介詞詞綴改某字己意義。以此手段,詞彙減得相當少。例如,有「good」(「好」)字,何必有「bad」(「壞」)字,橫㼭所需己意義俾「ungood」(「毋好」)表達得一樣感好,更好添。有一對自然相反己字箇時,惟要揀何字壓抑。譬如,任人點揀,將「unlight」(「毋光」)代「dark」(「暗」),或「undark」(「毋暗」)代「light」(「光」)。

現代話語法己第二特性爲其規律性。除著以下幾個例外,所有屈折變化一模一樣。如此,所有字己過去式同過去分詞一樣有「-ed」喺尾。「Steal」(「偷」)己過去式係「stealed」,「think」(「恁」)己係「thinked」,諸如此類,而「swam」「gave」「brought」「spoke」「taken」等形式革除著。所有複數加「-s」或「-es」。「Man」「ox」「life」尐複數係「mans」「oxes」「lifes」。形容詞比較級加「-er」「-est」(似「good」「gooder」「goodest」),毋規則己形式同「more」「most」壓抑著。

重可以毋規則感屈折變化己字,就係尐代名詞、關係代詞、指示代詞、助動詞。全部跟古老己使用,除著「whom」嫌多餘而廢著,同埋「shall」「should」己時態俾「will」「would」取代。構詞又有尐毋規則,俾說話迅速而暢順己需要引起。一個難咬己或易俾人聽錯己字根本造得毋好;因此,偶然會加幾個字母入一個字裏邊,或會保持古老己形式。但係爾個需要最主要喺乙詞彙之中有影響。\emph{點解}發音容易感重要,本文會之後說明。

乙詞彙:最主要由特登因政治動機而建造己字組成:毋單止個個字有政治上己暗示,而且會強加??desirable??己心理態度喺用者。毋明英社所有己原則,尐字會難用得啱。有時可譯成古代話,或甲詞彙中尐字添,但通常需要長己釋義,而不留都失去某尐意味。尐乙字係口頭語速記法,常挃成連串念頭落幾個音節,同時精確而有力過普通言語。

全部乙詞係複合詞,(「speakwrite」(「話寫」)爾類複合詞,當然甲詞彙搵得出,而此係爲方便己縮寫而已,無意識形態上己意味。)兩個以上己字或字己部份,擺埋一齊變易讀己形式。所產生己字不留都係動名詞,守普通己規律屈折變化。例:「goodthink」,大概「正統」義,又或者當名詞睇,「正統感恁」。屈折變化如下:動名詞,「goodthink」;過去式同過去分詞形式,「goodthinked」;現在分詞「good-thinking」;形容詞「goodthinkful」;副詞「goodthinkwise」;動詞性名詞「goodthinker」。

尐乙詞無跟任何詞源上己規劃而造。組成己字,由邊種詞類都有,任何次序都有,點改而令成個字易尐讀而保留其詞源都有。譬如,「crimethink」「(思罪)」有「think」「(恁)」字喺尾,而「thinkpol」「(思警)」擺佢喺頭,而第二個字「police」脫落著第二個音節。既然令聲音和諧己過程十分之困難,所以乙詞有多毋規律己??formation??過甲詞彙。譬如,「minitrue」「(真實部)」「minipax」(「和平部」)「miniluv」(「愛情部」)尐形容詞形式係「minitruthful」「minipeaceful」「minilovely」,只不過因為「-trueful」「-paxful」「-loveful」講得有尐麻煩。不過,原則上,所有乙詞可以一模一樣己規律屈折變化。

有尐乙詞尐意義十分微妙,未學好成個言語己人難得會明。例如,某份時報己頭版項目己常見句:「Oldthinkers unbellyfeel ingsoc」。至短己古代話翻譯:「Those whose ideas were formed before the Revolution cannot have a full emotional understanding of the principles of English Socialism」(「心性革命之前陶冶??ideas were formed??己人情感上毋能夠對英式社會主義有完整己瞭解」)。而爾個翻譯亦毋令人滿意。首先,爲著理解條以上己現代話引文己成個意味,要對「Ingsoc」(「英社」)有淸楚己概念先。尚且,對基本「英社」有良好己教授己人先識欣賞「bellyfeel」何其有力。「Bellyfeel」解一個盲目己狂熱己接受,而今難尐想象;又轉講「oldthink」,意義與邪惡同墮落混雜難分。而有尐現代話字己特別功能(「oldthink」卽其中一個),毋係表達一尐意思而係毀滅佢地。爾尐必然少己字尐意義,引伸到包含成棚字,以致廢除而毋記得,既然佢地俾一個全面己字形容足。現代話編者己最大困難,毋係創造新字,而係創造之後,搞淸楚佢地尐意義,卽係,搞淸楚邊尐字俾佢地己存在消滅著。

同以上講過己「free」字一樣,曾經含有異端己意義己詞,只不過??undesirable??己意義??purge (removing meaning)??著。無數己字,好似「honour」(「??honour??」)「justice」(「正義」)「morality」(「??morality??」)「internationalism」(「??internationalism??」)「democracy」(「民主」)「science」(「科學」)「religion」(「??religion??」),絕埋滅添。佢地俾幾個??blanket words??包含著,而??by covering them??,革除著。例如,所有關自由同平等己字,俾「crimethink」(「罪恁」)一個字包含,而所有關客觀同??rationalism??俾「oldthink」(「古恁」)包含。重明確就危險。黨員,要有一個似古代己猶太人己世界觀:所有其他國家拜尐假神。毋使知爾尐神叫做巴力、歐西里斯、摩洛、阿斯他錄等等:識少尐,應該對佢己觀念重好添。佢識耶和華同耶和華尐戒律:所以,叫第尐名己或有第尐特徵己神,卽係假神。同樣感,黨員知物也係良好行爲,而用非常含糊己言語,形容得到點樣出得軌。譬如,性生活俾「sexcrime」(「性罪」)「goodsex」(「好性」)兩個現代話字管理澌。罪性,任何??sexual misdeed??寫都包:和姦、通姦、同性戀、其他種變態,而且,??性交practiced for its own sake??。既然全部都??一樣equally culpable??,而原則上??punishable by death??,所以毋使逐個列出。由科學同技術己用辭組成己丙詞彙,或者需要改名畀某尐性??sexual aberration??,但係平民毋使識。佢知「好性」係物也意思——卽係話,平常己??coitus between man and wife夫妻性交??,生子係惟一個目的,而??no physical pleasure on the part of the woman??:其他屬於「性罪」。用現代話恁,??seldom possible to fllow a heretical thought further than the perception that it WAS heretical??:之後,需要己字毋存在。

乙詞彙無字意識形態上中立己。數多係??euphemism«??。爾尐字,譬如,「joycamp」(「快營」,勞改)、「Minipax」(「和平部」卽戰爭部)同表面上己意義相反。反而,有尐字露出一個坦白而鄙夷己見解對大洋洲己社會。例如,「prolefeed」(「無產者飼料」),卽其黨尐傳畀羣眾己垃圾娛樂同虛假新聞。其他字會兩性己,向其黨有「好」義,向敵人有「壞」義。而且,有好多字,似簡單己縮寫,意識形態上己意味毋來自其意義,而其結構。

全部有或可能有政治意義己字,盡量挃入乙詞彙。所有組織、團體、道義、國家、機構、公共建築,個個都改著名做熟識己形式;卽係,一個易讀、音節至少、保全本來己詞源己字。例如,溫斯頓己職場,眞實部己記錄部門,叫「recdep」(「記門」),小說部門叫「ficdep」(「說門」),節目部門叫「teldep」(「節門」),諸如此類。毋單止爲著慳時間己。連喺二十世紀早期,收得縮己字經已係政治術語己特徵;而有人發現,用爾種縮寫己傾向喺極權國家同極權組織至明顯,例如「Nazi」(「納稅」)「Gestapo」(「蓋世太保」)「Comintern」(「共產國際」)「agitprop」(「宣鼓」)「文革」。本身本能感採用,但係喺現代話之中有特登己目標。有人察覺,令一個名縮細時,意思同時會縮細而微妙感變,變得無著尐平常會扔住己聯想。譬如「無產階級文化大革命」,引起紅旗、喇叭、毛主席、兵荒馬亂、戰火紛飛、雞頭破壞己??composite image想象??。反而,「文革」只不過提起文化己變革,又或者甚無害己「文字改革」。「文革」差毋多毋使恁都講得出,而「無產階級文化大革命」一個詞至少會令人暫停一下。同樣感,「Minitrue」所引起己聯想少過「Ministry of Truth」所引起己。爾個毋單止解釋盡可能縮寫己習慣,又解釋埋誇張己細心向整每個字易拼。

喺現代話,除著意思精確,聲音和諧係最重要己考慮因素。語法規律性不留都有需就爲佢犧牲著,完全合理,既然爲著政治上己目的,最主要係字短促、意義無可能搞錯、講得快速、令講者回想得盡少。靠個個字都好相似,乙詞彙尐字更大力添。「Goodthink」「Minipax」「prolefeed」「sexcrime」「joycamp」「Ingsoc」「bellyfeel」「thinkpol」等字,一味兩三個音節,重讀喺最前同最後己音節之間分得平均,??encourage (behaviour)??一種一輪嘴己、又斷續又單調己??style of speech??,卽其黨己目的,計劃令說話,尤其是關任何意識形態上毋中立己說話,盡可能從意識分裂。日常生活之中,當然需要,或有時需要恁先講,但有人叫黨員做出政治或倫理上己判斷時,應該同機關槍一樣自動噴出正確己意見。佢己訓練適合感做,言語亦幾乎萬無一失,而尐字己結構,又刺耳,又含有跟英社己精神己一種故意己醜怪,喺爾方面更有幫助。

選擇極少又幫得。由我地己來睇,現代話己詞彙微細,不停感設計緊新己減細己方法。現代話同大多數言語毋同,詞彙毋會年年擴大,反而縮細。個個減少算係利潤,既然選擇己範圍越細,思念己誘惑力越弱。最終,有望淸楚己說話會就係感由喉頭流出來,同??higher brain centres??毋再有關。爾個目標喺現代話「duckspeak」(鴨話)一字,意義「似鴨感呱呱叫」,坦白感承認著。同數個乙詞彙字一樣,「duckspeak」意義兩性己,假如尐呱出己意見正統己,惟有稱讚,況且時報話其黨己演講者係「doubleplusgood duckspeaker」時,卽溫和而矜貴己恭維。

丙詞彙:補充其他詞彙,全包括科技術語。雖然似今陣用己科學術語,不過有用慣常己細心去令佢地意義明確同淸除??undesirable??己含義。依同前兩個詞彙一樣己語法。日常定政治上己說話,亦毋通用。一個科學家或技術專家從張尊登畀佢己職業己單裏邊搵得澌所需己字,而佢通常會剩係識滴感多第尐單尐字。好少字張張單有,而且無字表達以科學為心理習慣,或思考方法,無論邊個分支之中。乃至無字形容「科學」,既然所有其意義俾「英社」包足。

就前文所述可知,用現代話表達非正統己意見,到好低己程度以上,直頭無可能。當然講得好俗己異端說話,一種褻瀆。譬如,雖然講得「Big brother is ungood」(「大哥毋好」),但係喺正統己耳子之中,係只不過明顯己謬論,而且無可能有羅輯己理由,因為所需己字無得用。逆英社己念頭,剩係可以含糊而毋識形容感考慮,而剩係可以用十分廣泛己字提起,廣泛得包含成唔幾種異端,毋逐個指定,而一齊譴責。其實,惟有非法感譯某尐字變返古代話,現代話先有非正統己用途。例如,「All mans are equal」(「人皆平等」)算係合法己現代話句,只不過到「All men are redhaired」(「人皆紅髮」)算係合法己古代話句己程度為止。雖然無語法己錯誤,但係表達一個明顯己大話——全部人一樣高、一樣重或一樣好力。政治平等己概念已毋存在,所以爾個次要意義從「equal」??purge (remove a meaning)??著。一九八四年,古代話重係普通己??means of communication??箇時,理論上用現代話字箇時,會記起原本己意義己風險。實際上,避免感做對有良好己教授對基本「doublethink」(「雙恁」)己人無物困難,但係毋夠幾代人己時間之內連感己疏忽己可能性都消失埋。猶如未聽過象旗己人毋會識「象」同「士」尐次要意義,以現代話為惟一個言語長大己人毋會知「平等」以前會解政治平等,毋會知「自由」以前會解知識自由。有好多罪佢毋會能夠犯,好多錯毋會能夠出,只不過因為佢地無名,無辦法想象。將來可見,現代話尐特性會越來越突出——字越來越少,意義越來越嚴格,誤用己可能性一路減低。

古代話斷然取代著箇時,同過去己最後己聯繫會斷埋。歷史經已改造過,但係到處過去己文學己一尐片段有留存,審查得毋完整,人保持古代話己知識就讀得到。未來箇陣,就算重有留存,爾尐片段會理解毋到而毋譯得。無可能將某段古代話譯成現代話,除非佢關一個技術己過程或日常己動作,又或者經已有正統(「goodthinkful」就係現代話己講法)己傾向。事實上卽係話,一九六零年之前印刷己書毋會翻譯得澌。革命之前己文學剩係受到意識形態翻譯——卽係,意思同言語一樣要改。例如「獨立宣言」己著名一段:

	我地以爾尐真理為自明:人人生來平等,有造物主賦予己不可剝奪己權利,包括生活、自由、追求幸福己權利。政府為著保護爾尐權利而建立,持有受治理者所允許己權利。政府一想毀壞箇尐目的,人民有權改革或革除佢,建立新政府……

毫無可能譯佢變現代話而保持原本己意思,至多用「crimethink」一字吞著佢。全文翻譯剩惟可以係意識形態翻譯,傑佛遜尐字會譯成對極權政府己頌詞。

好多過去尐文學經已如此改造緊。由於威望己考慮,必要保持某尐歷史人物己記憶,同時令佢地尐偉績同英社一致。各種作家尐傑作翻譯緊,例如莎士比亞、米爾頓、斯威夫特、拜倫、狄更斯等等:完成之後,??original writings??,同所有過去尐殘餘,會毀滅著。爾尐翻譯係一單緩慢而己事幹,預計廿一世紀頭十年廿年先完成。又有好多實用而已己文學——不可缺少己技術手冊等等——要同樣感譯。首先為著畀尐時間畀初步己翻譯,最後採取現代話己日期定著二零五零年感遠。