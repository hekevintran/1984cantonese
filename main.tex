% Author: Quincy Lam (林立崑)
% Licensed under Creative Commons BY-NC-SA 3.0
\documentclass[a4paper,14pt]{memoir}
%No need to use -papersize
\usepackage{fontspec}
\setmainfont{Sun-ExtA}
\usepackage[fallback]{xeCJK}
\usepackage{rotating}
\usepackage{adjustbox}
\usepackage{verbatim}
\usepackage[normalem]{ulem}
\usepackage{lscape}
\usepackage{etoolbox}
\usepackage[backend=biber,style=authoryear]{biblatex}
\addbibresource{main.bib}
\newfontlanguage{Chinese}{CHN}

%%%%%%%%%%%%%%%%%%%%%%%%%%%%%%%%%%%%%%%%%%%%%%%%%%%% FONTS

%\xeCJKDeclareSubCJKBlock{puncts}{`。,`,,`、,`?,`!,`:,`;,`﹃,`﹄,`﹁,`﹂}
\xeCJKDeclareSubCJKBlock{puncts}{"FE10 -> "FE19, "FE30 -> "FE48}
\setCJKfamilyfont{zhvert}[puncts={[Vertical=RotatedGlyphs]{AR PL UMing HK}}, Vertical=RotatedGlyphs]{Sun-ExtA} 
\setCJKfallbackfamilyfont{zhvert}[Vertical=RotatedGlyphs]{Sun-ExtB}
\setCJKmainfont[puncts=AR PL UMing HK]{Sun-ExtA}
\setCJKfallbackfamilyfont{rm}[]{Sun-ExtB}

\setCJKfamilyfont{em}[ Vertical=RotatedGlyphs]{AR PL UKai HK}

%sbw = Sun-ExtB wrap
\newcommand*{\sbw}[1]{\setmainfont{Sun-ExtB}#1\setmainfont{Sun-ExtA}}


%。,、?!:;「」『』
%︐︑︒︓︔︕︖﹃﹄﹁﹂︵︶︽︾︙︱

%%%%%%%%%%%%%%%%%%%%%%%%%%%%%%%%%%%%%%%%%%%%%%%%%%%%% PAGE LAYOUT



\makepagestyle{myPage}{%
}
\makeevenhead{myPage}{}{\leftmark}{\thepage}
\makeoddhead{myPage}{\thepage}{\rightmark}{}
\makeevenfoot{myPage}{}{}{}
\makeoddfoot{myPage}{}{}{}


\makepsmarks{myPage}{
\def\partmark##1{\markboth{##1}{}} 
\def\chaptermark##1{\markright{##1}} 
}

% Chapter pages use the plain style, so we will change it in order to place the page numbers where I want them
\makeoddhead{plain}{\thepage}{}{}
\makeoddfoot{plain}{}{}{}


\makechapterstyle{myChapter}{%
	\def\chapterheadstart{\vspace*{\beforechapskip}}
	\def\chapnamefont {\CJKfamily{zhvert} \large}
	%\def\printchaptername{\chapnamefont \@chapapp}
	\def\printchaptername{}
	%\def\chapternamenum{\space}
	\def\chapternamenum{}
	%\def\printchapternum{\chapnumfont \thechapter}
	\def\printchapternum{}
	%\def\afterchapternum{\par\nobreak\vskip \midchapskip}
	\def\printchaptertitle##1{\chapnamefont ##1}
	%\def\afterchaptertitle{\par\nobreak\vskip \afterchapskip}

}


%For the part 1,2,3 pages
\renewcommand*{\printpartname}{}
\renewcommand*{\partnamenum}{}
\renewcommand*{\printpartnum}{}
\renewcommand*{\parttitlefont}{\Huge\CJKfamily{zhvert}}


%One day figure out how this works
%http://tex.stackexchange.com/questions/38593/rotating-text-by-90-degrees
\begin{comment}
\let\antilandscape\landscape
\let\endantilandscape\endlandscape
\def\LS@antirot{%
\setbox\@outputbox\vbox{\hbox{\rotatebox{-180}{\box\@outputbox}}}
}
\patchcmd{\antilandscape}{\LS@rot}{\LS@antirot}{}{}
\end{comment}

\makeatletter
\renewcommand*{\LS@rot}{%
  \setbox\@outputbox\vbox{\hbox{\rotatebox{-90}{\box\@outputbox}}}}
\makeatother

% A square of sidelength (1-x) consumes half of the area of a square of sidelength 1 if x=(2-sqrt(2))/2=0.29289321881345
% (1-x)=0.70710678118655, n=0.41421356237309 where (1+n)*(1-x)=1, n/2=0.20710678118655
\newcommand*{\fuse}[2]{{#1\raise0em\hbox{\scalebox{0.70710678118655}{\begin{adjustbox}{margin=0.20710678118655em 0em 0.20710678118655em 0em}{#2}\end{adjustbox}}}}}
\renewcommand*{\emph}[1]{{\CJKfamily{em}#1}}

\renewcommand*{\notenumintext}[1]{\hskip-1em\raise0.45em\hbox{\rotatebox{90}{\miniscule #1}}}
\makepagenote
\notepageref
\renewcommand*{\notesname}{註釋}
\renewcommand*{\notedivision}{\chapter{\CJKfamily{rm}\notesname}}
\renewcommand*{\printpageinnotes}[1]{(\pageref{#1}葉)}

\begin{document}

	\pagestyle{empty}
\setlength{\parindent}{0em}
{\HUGE 一九八四} \\ \\
{\large 佐治・奧威爾 著\\
林立崑    譯} \\ \\
\begin{vplace}[20]
\normalsize 本著作採用創用 CC 姓名標示-非商業性-相同方式分享 3.0 Unported 授權條款授權。
\end{vplace}
\cleardoublepage

	\pagestyle{myPage}
	\chapterstyle{myChapter}
	\punctstyle{quanjiao}
	\setlength{\parindent}{1em}

	\tableofcontents
	

	\begin{landscape}

	\CJKfamily{zhvert}
	\part[(一)]{︵一︶}
	\chapter{第一章}
	四月份又𥋇
\pagenote[𥋇]{音caang4。耀眼。}
又冷个一日,時鐘懕
\pagenote[懕]{音ngaam1。方纔。又妥當。《說文》:「安也。」《廣韻》:「一鹽切。」《朱註》:「厭厭,安也,亦久也。」《小戎傳》:「厭厭,安靜也。」《康熙字典·心部·十四》:「又通作厭。」《說文解字注》:「釋文及魏都賦注引韓詩愔愔,和悅之皃。」有非疑母字有j-聲母與ng-聲母雙粵音,多屬影、喻、日母(陳伯輝:1997)。}
敲響緊十三點,爲著
\pagenote[著]{音zo2。助詞,表示動作已完成。脫落韻尾之喉塞音而得今音(陳伯輝:1997)。}
擋刺骨个風吹,溫斯頓 · 斯滅將下爬貼住心口行,快快脆𤗈
\pagenote[𤗈]{音sip3。滑進某處。又插物入罅。}
入勝利大廈,快毋
\pagenote[毋]{音m4。不。}
過避免一陣塵跟彌
\pagenote[彌]{音maai4。助詞,表示同樣、補足、完畢。支韻字有讀aai,例如「徙」「舐」。}
佢
\pagenote[佢]{音keoi5。代詞,指第三人。又作「渠」「𠍲」。}
入去。

走廊有陣
\pagenote[陣]{音zan6。「陣味」詞中受「味」字影響,同化而轉韻尾,卒成為氣味之量詞。}
煲椰菜同
\pagenote[同]{與。}
舊地薦个
\pagenote[个]{音ge3。助詞,表示形容詞性、修飾、領屬、肯定語氣。亦代替所指示名詞。相當於國語「的」。}
味。一頭埲
\pagenote[埲]{音bung6。量詞。}
牆釘著張海報,大得懠
\pagenote[得懠]{音dak1 zai6。過份。}
喺
\pagenote[喺]{音hai2。於,在。}
室內掛,工子
\pagenote[工子]{音gung1 zai2。畫。又娃娃。}
得隻巨大个樣
\pagenote[樣]{音joeng2。面。又外貌。}
,一米闊有餘:係
%\pagenote[係]{音hai6。助詞,表示解釋。相當於國語「是」。}
個大概四十五歲个男人塊面,鬍鬚又黑又粗,容貌健壯英俊。溫斯頓向樓梯行,費事試下搭部𨋢
\pagenote[𨋢]{音lip1。昇降機。}
,橫㼭
\pagenote[㼭]{音dim6。豎。橫㼭,橫豎,反正。}
平時都用毋逳
\pagenote[逳]{音juk1。動。}
,而且今陣
\pagenote[今陣]{音gaa1 zan2。最近。gaa1卽「今下」之合音(陳伯輝:1997)。}
日頭無電力,準備仇恨禮拜个節儉運動个一部分。因為間單位要上七條樓梯先到,所以成三十九歲、右腳眼上邊生靜脈曲張性潰瘍个溫斯頓,要停低幾次敨
\pagenote[㪗]{音tau2。休息。又呼吸。}
下
\pagenote[下]{音haa5。助詞,表示短時間。}
。喺個個樓梯口,對住𨋢門,海報个大隻樣喺道注視住
\pagenote[住]{助詞,表示持續。}
。幅畫特登
\pagenote[特登]{故意。}
畫得好得意:人行一步,對眼好似會跟住人行。下邊標題寫住:\emph{大哥睇
\pagenote[睇]{音tai2。視。}
實
\pagenote[實]{助詞,表示密切。}
你}。

單位裏邊,一把甜美个聲喺道
\pagenote[喺道]{音hai2 dou6。正在。又於此。}
念出一連串點
\pagenote[點]{如何。從「何等物」省「何」得「等物」而合音。「等」古屬蒸韻。}
樣關
\pagenote[關]{關於。}
生鐵个數目。把聲從一塊長方形、似塊蒙
\pagenote[蒙]{音mung4。模糊,不光亮。《書·洪範傳》:「蒙,隂闇也。」}
个鏡甘
\pagenote[甘]{音gam2。如此。又連詞。又助詞,修飾以後之語。}
个鐵牌播出來,安著喺右手邊埲牆面。溫斯頓扭一粒紐,把聲細少少,不過尐
\pagenote[尐]{音di1。些。又助詞,表示某物比較如以前之語。}
字重分得出。部機器(所謂電幕),可以校
\pagenote[校]{音gaau3。調整。}
暗,但係無
\pagenote[無]{音mou5。不存在。}
辦法熄澌
\pagenote[澌]{音saai3。助詞,表示盡、全然。}
佢。溫斯頓去窗前邊望下自己:瘦蜢蜢、著住其黨件藍色蛤乸衣
\pagenote[蛤乸衣]{音gap3 naa2 ji1。工作服。蛤乸,蟾蜍。}
,令佢睇起上來
\pagenote[來]{音lai4。陳伯輝有提(1997)。}
重瘦。頭髮十分淺,面色天生紅潤个,皮被粗个番鹼
\pagenote[番鹼]{音faan1 gaan2。肥皂。}
、屈
\pagenote[屈]{音gwat6。鈍。又禿。《集韻》:「渠勿切。」《說文解字注》:「鈍筆曰掘筆。短頭船曰撅頭。皆字之假借也。」陳伯輝提過(1997)。}
个刀片、上年冬天个寒風癩
\pagenote[癩]{音laa2。刺激。《集韻》:「傷也。疥也。」疥,《說文》:「搔也。」}
到𩰶歰歰
\pagenote[𩰶歰歰]{音haai4 saap3 saap3。𩰶,歰,不滑。𩰶,《集韻》:「何開切,音孩。」《類篇》:「麧也。一曰糜中塊。」比較:糙,《廣韻》:「米糓雜。」}
。

由道閂
\pagenote[閂]{音saan1。關閉。}
著个窗望出去,出邊世界好似好冷淡。下邊街道,風喺道卷起尐塵同尐爛紙。日頭雖然曬住、天又藍得𥋇眼,但係物
\pagenote[物]{音mat1。何。陳伯輝提過(1997)。}
也
\pagenote[也]{音je5。物。}
都好似無物色甘,周街尐海報之外。箇
\pagenote[箇]{音go2。遠指示詞。彼。}
隻黑鬚僗
\pagenote[僗]{音lou2。傢伙。}
由個個居高臨下个角度望落來,連對面間屋都有一張。標題寫住:\emph{大哥睇實你},箇對眼深望住溫斯頓自己箇對。下邊街頭,第張海報,一個角扒
\pagenote[扒]{音mit1。撕。《集韻》筆別切,𠀤讀若分別之別。擘也,剖分也。}
開著,被
\pagenote[被]{音bei2。}
風吹到擺來擺去,斷續露出兩個字:\emph{英社}。好遠下,有部直昇機喺道穿過尐屋頂之間,好似隻烏蠅甘懸停一陣,然後側住一邊飛走。巡邏來个,道道窗都𥊙
\pagenote[𥊙]{音zong1。偷視。}
入去。但係其實尐巡邏都毋使驚。尐思警先至
\pagenote[先至]{僅僅。又省作先,又省作至。}
驚。

溫斯頓背後,部電幕繼續儑
\pagenote[儑]{音ngap1。胡說。}
三儑四關生鐵同彌第九個五年計劃个超額完成。電幕同時傳播同錄音个。所有溫斯頓整个聲會收到,除著極之低沉个耳語,而且,喺塊鐵牌个視野之內,亦會被人見到。當然,無辦法知邊個幾時喺道睇住你。無人知思警會幾常時或順住物也系統揀邊個來睇。有可能所有人幾時都一切望住澌添。總之,佢敵
\pagenote[敵]{音dei6。《說文》:「仇也。」仇,先指對方,後指敵人。《廣韻》:「匹也,當也,輩也,主也。」脫落韻尾得今音。}
隨時都可以揀你。係一個習慣——何止習慣,本能感——當任何聲都會被人聽到、當任何動作,除非黑暗箇時,都會被人檢查个生活。

溫斯頓背住部電幕,寧願安全尐;而佢夠知背脊都識得透風个。一公里之外,白色个高聳个眞實部,溫斯頓个工作場所,從邋邋遢遢
\pagenote[邋遢]{音laat6 taat3。不整潔。}
个風景凸出來。含有模糊个討厭,溫斯頓喺道恁
\pagenote[恁]{音nam2。思念。陳伯輝提過(1997)。}
:倫敦,戰地機場一號个最主要城市。戰地機場一號經已係大洋洲个第三最高人口个省。溫斯頓盡力試下記起細個箇陣尐回憶,確定倫敦係毋係不留都甘。係毋係不留都有一棟又一棟腐爛十九世紀式个屋、尐牆要用木板頂住、尐窗用紙板同彌尐屋頂用波紋鐵補住、尐圍牆𦖿
\pagenote[𦖿]{音dap1。垂下。陳伯輝提過(1997)。}
𦖿地个?重有尐炸過來个廢墟,尐灰塵飄下飄下,尐柳葉菜喺尐瓦渣上邊周圍生;又有尐被炸彈炸平著个大尐个地方,起著幾羣寒酸个成個雞籠樣个木个棚屋,箇尐都係毋係不留喺个?無用,佢記毋到:佢尐童年經已毋記得澌,除著一連串光芒芒个活人畫,後邊無背景,通常毋清毋楚。

眞實部,現代話(\emph{大洋洲个官方語言。語法同語源,請見結論。})叫眞子,同任何見到个也毋同得好交關。一座巨大白雪雪个石屎金字塔,旁邊有無數个騎樓,飆
\pagenote[飆]{音biu1。昇。}
上天空三百米高。由溫斯頓个位置,懨懨睇得出刻著喺面,其黨三句口號:
\begin{quote}\emph{
戰爭係和平\\
自由係奴役\\
無知係力量
}\end{quote}
聽講,眞實部地上有三千間房,地底有相當个??ramifications??。分散喺倫敦,重有三座相似而同樣大个建築物。喺佢敵下邊,隔籬尐樓變著好細粒,甚至由勝利大廈屋頂一望就望得到全四個。佢敵就係構成成
\pagenote[成]{音seng4。全部。}
個政治機構个四個部門尐大樓。眞實部,關注新聞、娛樂、教育、美術。和平部,關注戰爭。愛情部,保持法律秩序。大把部,負責經濟。現代話尐稱呼:眞子、和子、情子、大子。

愛情部先係好恐怖箇個,一道窗都無。愛情部,或半公里以外,溫斯頓從來未入過。係一個有本事先入得去己地方,而甘都要穿過一個滿鐵絲網、鐵門、隱蔽機場陣地个迷宮。連縛
\pagenote[縛]{音bok3。連接。}
住外牆个幾條街都有黑衫黑面个警衛,個個都抯
\pagenote[抯]{音zaa1。持。陳伯輝提過(1997)。}
支椎去巡查。

溫斯頓突然擰轉面。佢个表情經已變著冷靜个樂觀,最適合對住部電幕。佢穿過間房入細細間个廚房。甘早返
\pagenote[返]{音faan1。回歸。又助詞,表示恢復。}
屋其
\pagenote[屋其]{音uk1 kei2。家,住所。其,虛義。}
就犧牲著飯堂套餐,而佢知廚房一尐也食都無,除著一具
\pagenote[具]{音gau6。量詞。中古虞韻字多來自上古侯韻(陳伯輝:1997)。}
要留俾
\pagenote[俾]{音bei2。讓,許。又益。}
天朝早
\pagenote[天朝早]{音ting1 ziu1 zou2。明日晨早。ting1卽「天光」之合音(陳伯輝:1997)。}
食个黑麵包。佢由架上邊攞
\pagenote[攞]{音lo2。《集韻》:「揀也。」}
一樽透明个液體,簡簡單單个白色牌印住\emph{勝利氈}。陣味油到想嘔,好似米酒甘。溫斯頓斟差毋多成個茶杯甘多,鼓起勇氣,好似食藥甘吞著佢。

卽刻塊面紅澌,猛滮
\pagenote[滮]{音biu1。水由小孔出來。}
眼淚。口感好似飲硝酸甘,而且一吞,後尾䪴
\pagenote[後尾䪴]{音hau6 mei5 zam1。頭後部。}
就有被人用支橡膠棍攴
\pagenote[攴]{音bok1。陳伯輝有提(1997)。}
過來个感覺。不過,下一刻,肚裡邊个燒灼感覺散著,世界又好似歡樂著一尐。從一個䩌
\pagenote[䩌]{音caau4。有紋路。彭志銘提過(2009)。}
个印住\emph{勝利煙}个盒,攞著一支煙出來,毋覺意打直甘抯,俾尐煙草跌澌落地。第二支成功尐。佢返返入廳,喺電幕左邊張檯子坐低。由檯桶攞出來一個筆架、墨水同埋一本厚个空个四開本个書,後邊紅个,書皮有大理石纹彩。

毋知點解,電幕个位置好奇怪,無平常甘安喺埲??end wall??,俾佢睇到成間房,反而安著喺長个箇道,對住道窗。側邊有個淺个角落頭,溫斯頓坐緊
\pagenote[緊]{音gan2。助詞,表示繼續。}
喺道,建立箇時應該整來擺書架个。溫斯頓一坐喺爾個角落頭裏邊,挨得好後,睇起想來,電幕就睇毋到佢了
\pagenote[了]{音laa3。於問題,變陽平聲。}
。當然重俾佢聽得到,但係如果佢留喺而今
\pagenote[而今]{音ji4 gaa1。}
个位置,就睇毋到。某種程度上係間房个古怪个結構令佢恁起佢將做个也。

但係啱攞出來箇本書又有令佢恁起。一本特別令
\pagenote[令]{音leng3。漂亮。}
个書來个。奶白色个開始發黃个書頁,用或者四十年多無再製造个紙造个,而溫斯頓認爲本書應該老好多。佢喺某個貧民區(邊一區又毋記得著)个一座??frowsy little??古董鋪
\pagenote[子]{音zai2。陳伯輝有提(1997)。}
見到,卽刻渴望買著佢。雖然黨員毋應該入普通鋪頭(所謂「自由市場交易」),其實??rule was not strictly kept??,因爲好多樣也無第尐方法去攞,例如鞋帶同刀片。佢兩頭望著一下先,就𤗈入去,畀
\pagenote[畀]{音bei2。給,予。}
著兩個半,買著佢。當時未覺得有物也原因去買。佢鬼鬼鼠鼠甘喺個篋
\pagenote[篋]{音gip1。陳伯輝有提(1997)。}
裏邊拎
\pagenote[拎]{音ling1。持。}
著返屋其。雖然空个,但係都係一個有罪个所有物。

佢將做个也,就係開始寫日記。雖然毋犯法(無犯法也,因爲經已無法律了),但係假如被人敵知,肯定會被人判處死刑,毋係就起碼罰廿五年勞改。溫斯頓安個筆嘴喺筆架道,啜住佢來整𠞉
\pagenote[𠞉]{音lat1。脫落。}
尐油面。筆係古老个器具,簽名都少見。溫斯頓偷偷地、有少少困難感,攞得到,剩係因爲佢覺得甘
\pagenote[甘]{音gam3。如此,甚。}
令个奶白色个紙應該用支眞筆來寫,靡
\pagenote[靡]{音mai5。莫,勿。}
用個普遍个墨筆。其實佢毋太習慣親手寫也。除著好短个便條,通常物也都對住部話寫講,當然爾個情況毋得。佢沾
\pagenote[沾]{音dim2。將物投入液體。}
支筆喺樽墨裏邊,就猶豫著一秒鐘。??a tremor had gone through his bowels??。一點張紙就??decisive act??。好細、好馬虎甘寫:
\begin{quote}\emph{
一九八四年四月四日
}\end{quote}
佢挨返後。佢突然間覺得完全無助。首先,佢一尐都毋肯定今年係一九八四年。佢知個日子差毋多,因爲佢都幾知佢三十九歲,同彌相信佢一九四四或四五年出世个;而今陣,日子無可能準確過一兩年以內。

忽然間恁起,究竟,爾本日記係寫俾邊個个呢?俾未來,俾未出生个。佢考慮一陣紙上邊个可疑个日子,就撞到箇個現代話字「雙恁」。佢終於明白所做个也个嚴重性。究竟點樣同未來溝通呢?自然無可能。抑係未來會似而今,甘毋會聽佢話;抑係毋會似而今,甘佢而今个困境會對佢敵無意思。

獃著喺道幾耐下,望住張紙。部電幕轉著播刺耳个軍歌。佢覺得好怪,毋單止似乎失著表達自己个能力,甚至想寫物也都毋記得彌。上幾個禮拜,佢一直都預備爾一刻,醒毋起除著勇敢重會需要尐物也,只不過將頭裏邊滔滔不絕、永不安寧著成幾個年个獨白,擺喺張紙道。但係爾一刻連個獨白都停埋。兼且,佢个潰瘍開始痕得好緊要。佢毋敢擾
\pagenote[擾]{音ngaau1。搔,撓。上述j-聲母與ng-聲母雙粵音理論。口語變陰平聲。}
,因爲佢一擾就腫起。一秒又過一秒。佢剩係留意前邊張白紙、腳眼上邊个痕、巴巴閉个音樂、尐氈引起个小小醉感。

突然間佢慌失失博命寫,毋係太明自己寫緊尐物也。佢个又細又似細僗子
\pagenote[細僗子]{音sai3 lou6 zai2。兒童。}
个字跡十分之潦草,一路寫一路逐漸退化,以至句號都無彌:
\begin{quote}
一九八四年四月四日。昨晚
\pagenote[昨晚]{音cam4 maan5。cam4卽「昨暝」之合音(陳伯輝:1997)。}
去著睇戲。全部係戰爭片。有一套好正个講一個載滿難民个船喺地中海邊道被人炸到。觀眾覺得好搞笑見到個肥僗游緊水被直昇機追住,先見佢成隻海豚甘喺水道轆
\pagenote[轆]{滚動。}
,然後經過直昇機个瞄準線望佢,甘射到佢穿澌竉尐水染澌紅色就突然間沈著好似尐竉入著水甘,沈箇時觀眾哄堂大笑。甘就見到個救生艇裝滿細僗子,有個中年女人(可能係猶太人)坐喺船頭抱住大約三歲个細僗。細僗驚到喊挃個頭喺女人尐胸之間似乎想蜎
\pagenote[蜎]{音gyun1。爬。《分韻撮要》:「蟲行貌。」}
入佢甘女人攬實佢安慰佢雖然佢自己驚到面都靑微一路盡可能遮住個細僗似乎以爲擋得到尐指彈。直昇機抌個廿公斤炸彈落去佢敵道閃十分光芒艇得廢柴。甘影得好精彩一個細僗子隻手飛飛飛上天肯定前邊有錄影機个直昇機跟著佢上黨員尐位道就好大掌聲但係喺戲院个無產箇邊有女人無端端發爛謯
\pagenote[發讕謯]{音faat3 laan6 zaa2。抵賴。讕,《說文》:「怟讕也。」謯,《廣韻》:「謯訝訶皃。」}
話佢地毋應該喺細僗子面前播佢地毋毋道德毋好喺細僗前毋道警察鏟鏟著佢走無也卦佢話之尐無產點講尐無產先會甘無聊佢敵不留都
\end{quote}
溫斯頓停著,原因一部份係佢抽筋。佢毋知物也令到佢嘔一連串甘个垃圾出來。但係奇怪个就係,寫緊箇時,同時記得起好淸楚一個完全毋同个回憶,淸楚到就來想寫低。佢意識到,原來因爲爾個事件,佢今日突然間返來屋其開始本日記。

如果甘朦朧个事情眞係有發生个話,就喺箇日部門朝早發生。

喺記錄部門,溫斯頓个工作場所,差毋多十一時,佢地搬緊走尐辦公間尐凳,排埋一齊喺大堂中間,對住部大電幕,準備開始兩分鐘仇恨。溫斯頓啱啱喺中間某一排坐緊低,突然間見到兩個人入來,??knew by sight but had never spoken to??。一個係喺走廊常時旁邊經過个個女子。名毋知,但係知佢喺小說部門做也。既然成日見到佢手油个、抯住支士巴拿个,或者有份關尐寫小說个機器个機械工。睇起想來膽大、大概廿七歲、頭髮厚、樣有雀斑、行動??switft, athletic??。一條窄个鮮紅色腰帶,靑年反性聯合會个標誌,丩
\pagenote[丩]{音kiu5。纏繞。}
住件蛤乸衣个腰部幾次,啱啱夠緊??帶出bring out??佢條腰有幾??shapely??。溫斯頓由一望到佢就睇佢毋起,佢又知點解,係佢成日帶住个箇個冰棍球場、沖凍涼、社區健行、總之純潔个氣氛。溫斯頓差唔多全部女人都睇毋起,尤其是又令又後生箇尐。不留都係尐女人,尤其尐後生个,先至係其黨最盲信个、口號吞澌落肚个、屎友間諜、保正統个臥底。但係爾個女子引起一個更危險个印𧰼。有一次佢地兩個經過箇陣,佢側下個頭望一望佢,似乎睇透溫斯頓感樣,令到佢滿心黑色恐怖。都恁過佢係思警員毋定。其實,少少誇張个。但幾時喺佢附近,佢都仍然覺得一個古怪个心慌,溝埋恐懼同埋怨恨。

另一個係位男人叫奧巴仁,一個內黨員,佢份工超標重要同遙遠到溫斯頓明白得好少。大家一見到一個內黨員件黑色蛤乸衣就靜著一陣。奧巴仁,大隻、頸粗、面兇得有尐搞笑,雖然感佢有佢自己種魅力。佢一整一整佢幅眼鏡,就古怪感會消除懷疑,十分細微一個小動作——某個毋釋得明个方面上,一種古怪个斯文。如果今陣重有人感恁个話,就覺得佢似一位遞出佢个鼻煙盒个十八世紀紳士。喺十二年之間,溫斯頓可能有見到佢感多次。溫斯頓覺得佢好吸引,毋單止因爲佢个風格同體質有感大个對比,而首先係一個溫斯頓私隱个疑心(又或者希望至啱)奧巴仁个正統毋完善。一睇佢塊樣就有感個感覺。當然,或者佢个表情毋係非正統,精伶而已。點都好,都似一個謦得欬个人假如搵到方法避開部電幕。溫斯頓完全未試過查實佢个意見,又無機會感做。而今,奧巴仁就望一望佢個手錶,發現就來十一點,顯然選擇留低喺紀錄部門過埋兩分鐘仇恨先。佢揀個位同溫斯頓一行,隔幾個位。一位細粒沙色頭髮个女人坐喺佢地之間。深色頭髮个女子坐溫斯頓後邊。

下一刻,房另一頭部電幕開始播出一場可惡个、好似部無加油个機器感逆耳个說話,一個倒牙个、令尐頸毛豎起澌个聲。仇恨開始著。

照舊,伊曼紐 · 葛士田,人民公敵,登出來。到處觀眾喺道噓。箇個金毛女人又驚又惱到吱一聲。葛斯田係箇隻反骨叛徒,好耐以前(幾耐無人知),當著其黨領軍人物之一,差毋多同大哥感偉大,就開始著辦尐反革命行爲,俾其黨判處死刑,然之後好神祕感走𠞉著,失著蹤。兩分鐘仇恨个節目日日都變,但係無一個葛士田毋係最主要个人物。背叛之一、其黨最初个染污,所有以後反黨行爲、所有背叛、怠工、異端、偏離,出自由葛士田尐邪教。不知何處,佢繼續生,繼續圖謀:或者喺海外,喺佢尐外國主計官个保護之下,或者(時時一句是非)重喺大洋洲裏面。

溫斯頓個隔膜縮緊著。佢無可能見到葛士田个樣而毋覺得一個鋒利个一混感情。瘦个猶太樣、一頭白色毛毛、細細個羊咩鬚,睇上來一個好醒个樣,而毋知點解又賤格,鼻長得癡癡地,尖托住副眼鏡。似隻羊感个樣,把聲又有羊个性質。葛士田又喺道䛅佢个狠毒个攻擊對其黨个學說,誇張同變態到細僗哥都應該睇得出,而啱啱夠似眞到令自己驚人地無感淸醒,可能會信佢。佢喺道侮辱佢,喺道詰責其黨个獨裁,喺道問取同歐亞洲个和平卽刻結束,喺道提倡言論自由、新聞自由、集會自由、思想自由,佢喺道猛嗌革命出賣著,全部一流又一流感滮,整蠱其黨時常時个講法,現代話个辭彙都有,的確多過一個黨員會平常用个。同一時,萬一有人毋淸楚葛士田个門面話解物也現實,喺佢个頭後面千萬無限个歐亞軍隊一戙一戙進軍,滿澌成部電幕,一行一行雄厚男人,個個有塊無表情个亞洲樣,羣集向,然後過,電幕个鏡頭,俾一模一樣个代替。腳步个鼓點做葛士田狂叫个氣氛。

仇恨未到三十秒鐘,人一半經已喺道鬥氣大聲嗌。得戚个羊樣,同埋後面歐亞軍个威力,實在受毋住,一恁起或見到葛士田都經已會令人又惱又驚。人憎佢重多過憎歐亞定東亞,反正大洋洲同一個爾尐大國打緊仗个時候,通常都同另一個和平相處。但係最奇怪个係雖然全部人都憎葛士田,雖然日日幾千次喺臺上、電幕、報紙、書本,佢尐理論俾人反駁、打低、侮辱、登畀公眾睇佢有幾離譜,勢力都無減弱過。不留都有新个豬頭丙等緊俾佢𧨾。無一日丙去思警毋會揭穿葛士田指令个間諜同怠工者。佢指揮一個又巨大又奧妙个叛軍,一個地下起義串謀者个網絡,風聞話叫兄弟會。又有嘴逳,有一本好得人驚个書,收集所有異端,葛士田寫个,爾處箇處靜靜雞傳來傳去。無名个,人剩叫佢做,如果敢提个話,「箇本書」。但人係剩聽是非八卦至知啫。可以避免个話,普通黨員毋使䛅兄弟會或箇本書就好啦。

到第二分鐘仇恨變激烈。人喺位道跳跳紮、大聲喊叫,爲著停止箇把聲喺佢地个頭裏面咩。箇個細粒沙毛女成面紅澌,把口係道開開閂閂似條離水魚。奧巴仁个大隻樣都紅,喺張凳道坐得好直,身一路震感呼氣,猶如佢要對住波浪徛穩。後邊深色頭髮個女子開始尖叫:「你死豬!爛豬!臭豬!」,忽然間攞著重个一本現代話字典,𠌸著佢去電幕道。撞到葛士田个鼻哥就彈返落地,把聲滔滔不絕感繼續。溫斯頓突然醒起,原來佢都同佢地一齊叫,暴力感踢住凳腳。兩分鐘仇恨最惨个毋係必須參加,而係無可能毋參加,三十秒鐘裏就扮都毋使扮。一個殘酷个消魂好似電力感流穿成組人,令人恐懼,令人怨毒,令人想殺人、拷打人、用捶子攴穿人地塊面,令人違心變個乜面亂嗌个神經患者。但係個人覺得个仇恨係一個抽𧰼、無方向个感情,點蠟燭感易可以調佢去另一個主題。所以,一刻溫斯頓个仇恨一尐都毋係對葛士田,反而對大哥、其黨、思警,此時佢侹佢傷心,爾個孤獨、成日俾人笑个異教徒,眞實正經个獨一個守護者喺爾個大話世界。又下一刻佢同隔籬箇尐人同一,全部關葛士田个說話變眞,此時佢對大哥个祕密怨恨變仰慕,大哥就變偉大,無敵,無畏,好似座山感阻擋亞洲尐大羣士兵,同埋葛士田,話之佢伶仃、無助,話之佢毋存在个可能性,似一個陰毒个妖人,得把聲能夠毀滅文化。

有時,自動轉個人个仇恨向爾樣箇樣都可以。忽然間,好似發完惡夢彈起身感衝動一個動作,溫斯頓成功調佢个仇恨由電幕塊樣向後邊个深色頭髮女子,就見到精彩、美麗个迷幻。佢會用支橡警棍攴死佢。佢會綁住佢个赤身喺條柱道,好似聖巴斯弟盎感射穿佢一拃箭。佢會強姦佢,達高潮就斬開佢條頸。今次最正个,佢終於明白佢點解感憎佢。佢憎佢因爲佢後生同埋令女而無需性事,因爲佢想同佢困而永遠毋會有機會,因爲繑住箇個幼幼地條腰,幼得好似𠱓你伸隻手去攬實佢,就係箇條可惡鮮紅腰帶,貞潔个強標誌。

仇恨就達高潮。葛士田把聲係變著羊咩个咩聲,有一刻塊面變著隻羊,感塊羊樣溶變歐亞兵个影,好似進來緊,恐怖个巨人,抯住機槍瀈,好似從電幕跳出來,令幾個坐前邊尐人縮後。但係,敵人就變著大哥,黑頭髮,黑鬍鬚,滿面力量同沈靜,面大隻到就來滿澌成個鏡頭。無人聽得到大哥講物也,幾句鼓勵,喺戰中䛅个也,分毋出個個字,但係見到人講就補返個人信心。感大哥个樣就消失,其黨三句口號粗體出現:
	
戰爭係和平
自由係奴役
無知係力量

但係人人都重覺得見到大哥个樣喺部電幕道,好似留著個印𧰼喺佢地个眼珠道,過著一陣先彈返起。個細粒沙毛女人趴著喺前邊張凳个挨屛道。震住䛅條句好似「我个救主!」,伸出佢對手向電幕,然之後篢住佢塊面,顯然喺道祈禱。

爾一刻成組人開始又低沈又緩慢感吟:「哥...哥!...哥...哥!」又講又再講,慢慢,第一同第二個「哥」之間有好長个停頓,一個沈重个和諧,有少少野蠻,聽到赤𨂽腳打鼓感滯。繼續著或者三十秒鐘都有。非常感動个時候成日聽到个一首歌謠。一部份係尊敬大哥个智慧同偉大,而更加係種自我催眠,用鼓點打低自己个意識。溫斯頓覺得佢个腸胃結冰。兩分鐘仇恨箇時佢忍毋住同大家癲,但係爾場無人性感吟「哥...哥!...哥...哥!」次次都令佢恐怖。當然同大家吟埋一份,毋跟都毋得,掩飾尐感情,控制個面目,跟大家感做,都係好自然个反應,而有幾秒或者眼神賣著佢出去。就爾一刻發生,眞係有發生个話。

同奧巴仁眼光相接著一陣間。奧巴仁徛緊向道,帶返對眼鏡箇個小動作之中。但係一秒半秒佢地眼光相接,箇一刻溫斯頓就知(眞係知!)奧巴仁同佢恁得一樣。係個搞毋得錯个傳情,似乎佢地兩個个頭腦變透明,思想一個去第個穿佢地尐眼感流。「我同你同意,」奧巴仁好似同佢感講感。「我明白澌你所覺得。我明白澌你个輕視、你个仇恨、你个嫌惡。但係毋使驚,我喺你身邊!」感就箇個曉得從佢塊面消失,面情變返大家感不可測。

係感多。溫斯頓經已懷疑無發生過。爾尐節目不留都無後段,剩養佢个信念,或者希望,有第尐人好似佢一樣係其黨个仇人。或者宏大个地下串謀尐風聲係眞个,或者兄弟會眞係有!雖然不斷有拘捕、自認、正法,都無可能知兄弟會毋係剩係一個故子,有時佢信,有時毋信,物證據都無,惟有眼可能睇到尐也,可能有無意思,抑係竊聽到幾個字,抑係廁所埲牆輕輕畫著一兩也,又有一次見到兩個陌生人撞到,一個細細个手勢,可能表達認識。全部齋估,應該全部喺佢想𧰼中只。佢返返佢个小房間無再望過奧巴仁,同佢跟進差毋多無恁過。恁到點做都會十分危險。一兩秒,佢地彼此望一望,就完啦。但係感都算値得記念,爾個孤獨監獄中。

溫斯頓醒返少少,坐直尐,呃一聲,尐氈昇返上來。

佢對眼又望實張紙,發現原來佢坐喺道考慮箇陣,佢又喺都好似自動感寫也,又毋係之前箇尐擠逼、醜怪个字,佢支筆一跣,跣過張字,寫著一行又一行:

	打倒大哥打倒大哥打倒大哥
	打倒大哥打倒大哥打倒大哥
	打倒大哥打倒大哥打倒大哥
	打倒大哥打倒大哥打倒大哥
	打倒大哥打倒大哥打倒大哥

再寫再寫,寫著滿半頁。

佢忍毋住覺得有尐心慌慌。都罷啦,反正寫箇尐字又無開本日記更加危險,但係有一陣間佢驚到好想撕開澌染污著箇尐頁,放棄成件事。

不過佢無感做,因爲佢知係經已無用。寫「打倒大哥」,抑係毋寫,都無分別,繼續寫,毋繼續,無分別,思警一樣會拐佢。佢犯著(物都無寫都犯著)基本、含所有其他犯罪个罪,所謂「思罪」。思罪毋係一個永遠都收得埋个也,或者可以避到幾何,幾年都可能,但係遲早都捉到你。

不留喺夜晚,不留尐拘捕夜晚發生。突然醒起身,粗暴个手搖住個膊頭,好嚴尐面圍住張牀。大多數都無審訊,無報告,人剩係感消失著,次次都夜晚。名從戶籍道挽著,個個做過尐也尐記錄擦著,存在俾人否認,然後完全毋記得,俾人地革除著,消滅著,常話「揮發著」。

佢又突然驚起,又開始猛亂寫也:

佢地會射我話之佢佢地會射我頸後面話之佢打倒大哥佢不留都會射你頸後面話之佢打倒大哥

佢覺得有少少羞恥,坐返佢張凳道,放低枝筆。下一刻,佢嚇到彈起。有人敲門。

感快脆!佢坐得好似隻老鼠感靜,死都希望邊個試著一次之後就會放棄。無今好彩,門繼續敲。最衰就會係拖長佢㗎啦。心雖然打鼓感跳,但塊面,依住長久个習慣,應該都無表情。起身,慢慢向道門行。


	\chapter{第二章}
	二
放緊隻手埋去個門柄時,溫斯頓發現漏著本開己日記喺檯上邊。「打倒大哥」周圍都係,字體爭尐由房另一頭睇得到,直情戇居。原來喺佢驚慌己時,都毋捨得墨重溼而閂本書,揩到尐字。

吸啖氣就開道門,卽刻冷靜好多。一位普通、面皺、頭髮疏己女人徛著喺出邊。

「啊,同志!」沈悶感喺道𠱓。「我都聽到你返著來,你恁你可毋可以過來睇我地己升盤啊?塞著啊,同埋——」

係巴臣士太太,同樓住己鄰居己老婆,(「太太」係一個俾其黨少少毋贊成己辭,係應該人人都叫「同志」,但係同有尐女人就自然得滯。)大約三十歲,但睇起想來老好多。覺得似乎有塵喺塊面尐痕裏面。溫斯頓跟住佢行。爾尐業餘修理差毋多日日都來煩佢。勝利大廈尐樓舊,一九三十年到建築,而今惡劣失修,天花板同埲牆成日𠞉灰塵,水喉次次冷天都爆,落雪屋頂漏水,暖氣機經已平常剩著半火,爲著慳電箇時完全熄澌。除非自做,修理要取委員會允許,整塊窗都可能要等兩年。

「當然剩係因爲阿湯毋喺屋徛啫。」巴臣士太太呆滯感話。

巴臣士單位大過溫斯頓箇間,有個毋同種亂。物也都有個踩過澌己樣,似乎啱有隻野獸來過感。運動己用具(冰棍、卜成手襪、爆著己足球、反著出來對短褲)鋪澌喺地下,檯面有一疊污糟碟同折著角己寫字簿。喺埲牆掛著兩張鮮紅旗,一個靑團己,一個間諜隊己,重有張全尺寸己大哥海報。都有箇陣煲椰菜味,但更加聞到係一陣汗除,一聞到就知(難解釋點解)係一個正在毋喺到己人己味。第間房有人抯住俾塊廁紙包住己把梳,喺道跟部電幕重播緊箇首軍歌一齊吹。

「係尐細僗啊。」巴臣士太太話,一尐毋安感望一望道門。「佢地今日未出過去,同埋又係...」

佢不留都會講著一半毋講埋尾箇段。廚房升盤裝住尐靑色污糟己水,差毋多裝到滿,椰菜味更大陣。溫斯頓𠌥低,檢查條喉己角。佢憎用佢尐手,又憎貓低因爲容易令佢開始咳。巴臣士太太手𠌸𠌸感睇住佢。

「當然如果阿湯喺屋徛己話,就一搞就搞得掂啦。」佢喺道講。「感樣己也物都鍾意己,眞係好靈活己佢,阿湯。」

巴臣士係溫斯頓喺眞實部己同事。肥肥地而多運動己男人,蠢到癡肺,一大匛弱智熱心,一個箇尐無異議、專注澌己驢子,其黨己穩定性重靠得佢地多過靠思警。啱三十五歲先從靑團毋願意俾人拋著,重有,陞靑團之前,佢搵到辦法留喺間諜隊多一年。喺部門,佢做毋使幾聰明己低能工,反而喺運動委員會,而且全部舉辦社區健行、自發示威、儲錢活動、物也義工己委員會,佢都係個領軍人物。吸著啖煙之後,佢會寧靜而驕傲感同你提起過著爾四年佢喺社區中心晚晚都出過場。一陣臭崩崩己汗味周圍都跟埋佢,走著都重喺道,證明生活得幾辛苦。

溫斯頓搞緊水喉角己螺帽,問:「你地有無支士巴拿?」

「支士巴拿啊?」巴臣士太太答,卽刻無澌骨氣。「我知我毋知啦。或者尐細僗——」

吹著多一聲梳就尐細僗子乒乓感走入廳。巴臣士太太帶返支士巴拿來。溫斯頓畀尐水流著出來,黑面感將個塞住條喉匛頭髮撩著出來,盡量洗佢尐手指喺尐凍水喉水,就返返去第間房。

「舉手啊!」兇惡把聲叫。

一個令子、睇上來硬淨己九歲細僗哥從張檯後邊彈著出來,抯住支玩具自動步槍威脅佢,細僗妹,大概細佢兩年,作同己手勢用塊木碎。兩個都著住藍短褲、灰裇衫、紅頸巾,卽間諜隊裝。溫斯頓舉佢對手過佢己頭,但係有少少心慌慌,個細僗哥嗌得感狠毒,毋係得個遊戲。

「你係叛徒!」細僗子嗌。「思罪犯!歐亞間諜!我射你!揮發你!抌你去鹽礦!」

忽然間佢地兩個圍住佢紮紮跳,「間諜!」同「思罪犯!」,細僗妹完全跟住佢阿哥。有少少驚,好似望尐老虎子喺道鬥,明知佢地好快變食人獸。有種狡猾己兇狠喺細僗哥己眼神裏面,顯然想攴或踢溫斯頓,知佢就來大到可以。溫斯頓恁:「好彩毋係眞支槍啫。」

巴臣士太太好緊張兩邊望,望溫斯頓,跟住尐細僗,又返返去。喺廳燈光好尐,溫斯頓用心睇到,原來眞係有塵喺佢尐面痕道。

「佢地眞係會嘈己啫。」佢話。「佢地失望因爲佢地無得睇場吊頸己。我忙得滯,阿湯放工都太夜啦。」

「做物我地毋可以去睇吊頸」細僗哥大聲吠。

「想睇吊頸!想睇吊頸!」細僗妹喺道講,重周圍跳。

溫斯頓記得,有尐歐亞戰俘,犯過戰爭罪,箇晚喺公園吊頸。大約每月有,人好鍾意去睇。尐寧子成日嘈想帶埋佢地去睇。溫斯頓同巴臣士太太拜著別就向道門行。但喺條通道無行六步,就佢己頸後面俾物也打得非常痛苦,猶如插著佢一條熾熱鐵線。佢掉轉面,啱睇到巴士臣太太挽佢己子返入門口。個細僗喺道挃佢己彈弓落袴袋。

「葛士田!」細僗喝,道門就閂著。但係最嚇到溫斯頓,就係個女人塊灰色面己表情:驚到無奈。

返返自己己單位,衝過部電幕,喺檯子道坐低,重捽緊條頸。電幕停著播音樂,而今,一個精簡、粗糙把聲,含有尐樂趣,描述緊新、啱啱喺冰島同法羅羣島之間起己浮動要塞尐所有武器。

有感己細僗,溫斯頓恁,箇個可憐己女人肯定生活得好得人驚。多一兩年,佢地會朝夕睇實佢有物也非正統。今陣,爭尐全部細僗子都感討厭。最惨己係尐組織譬如間諜隊逐個逐個轉佢地變成湊毋起己百厭子,但係一個對其黨想作反己都無,反而,佢地愛慕大哥同埋所有有關己,尐歌、遊行、旗子、健行、練玩具槍、大嗌口號、崇拜大哥,全部剩係個彪炳己遊戲。全部佢地尐殘暴向出邊,向其黨己仇人,向外國人、背叛者、怠工者、思罪犯。老過三十歲己人,好普通感會怕佢地自己己細僗子,又無點誇張,罕有禮拜時報無登一段寫個死多事子(多叫「細僗英雄」)偷聽到一句毋啱聽己話,告著畀思警。

條頸毋痛啦。求求其其執起枝,考慮重搵毋搵到多尐也來寫,忽然間有恁起奧巴仁。

幾年前(幾多年?肯定七年)發夢行間漆黑一片己房,過著人坐喺側邊,個人話:「無黑暗己地方見。」講得好靜,差毋多求其䛅,一個聲名,非命令。佢無停過繼續行。最奇怪己,箇陣喺夢裏面,爾句無物影響佢,惟係晏尐,慢慢先染上意思。而今毋記得係有著個夢之前定之後至初見都奧巴仁,又毋記得幾時先發現把聲係奧巴仁己。點都好,有爾個鑑定,係奧巴仁黑麻麻同佢講也。

溫斯頓不留都未肯定,今朝己眼色之後,都無可能知奧巴人係友人定敵人,又毋似大物也,佢地兩個互相理解,重要過友善或黨派。「無黑暗己地方見。」佢話。溫斯頓毋知係解物也,剩知點樣都會發生。

電幕把聲停著。響亮己喇叭聲入空氣遲滯己單位。把聲刺耳感繼續:

「各位觀眾,請留意!緊急消息啱啱由馬拉巴前線來到。喺南印度己士兵獲得一個光榮己贏利。我有權佈告,今日戰爭己結束變著更加近,消息就係感——」

壞消息來緊啦,溫斯頓恁。果然,跟住一個暴力己描述歐亞軍己滅亡,包埋誇張己俘虜同死亡數,就宣佈,下個禮拜開始,朱古力己限額由三十克減著去廿克。

溫斯頓又呃一聲。氈己影響開始退化,剩有個放氣己感覺。或者爲著慶祝,或者爲著浸去朱古力己掛念,電幕大聲播「爲你,大洋」。應該立正己。但係,喺而今己位置,無人睇到佢。

「爲你,大洋」俾輕鬆尐己音樂替代。溫斯頓行跟塊窗,背脊對住電幕。天氣仍然晴同凍。幾遠下,個導彈發個低沈己音感爆炸。爾輪個個禮拜大概廿個三十個跌落倫敦。

喺街道,張爛海報俾風吹到擺來擺去,「英社」爾個字又出又毋見。英社。英社尐崇敬己道義。現代話、雙恁、過去己易變性。佢覺得自己好似喺海底己森林游,一個巨人蕩失著喺個巨大己世界。孤獨。過去經已逝世,未來不可思議。有物把握人類之間有一個支持佢?又點知其黨己管治毋會永遠持續?如回答佢,記起眞實部己白色面己三句口號:
	
戰爭係和平
自由係奴役
無知係力量

從袴袋攞出來個兩毫半子,箇道,都有寫得細而淸楚,銀另一面有大哥己頭。喺個銀道尐眼照跟住你,錢銀、郵票、書己封面、旗、海報、同埋煙己包裝,卽到處。成日俾對眼睇住,把聲圍住。困定醒,打工定食也,入邊定出邊,沖涼定喺牀道,走毋𠞉。無也係私隱,除著頭殼裏面幾多。

太陽逳著,毋再照住眞實部尐窗,令佢變得堡壘己槍眼感冷酷。望住個巨金子形,心裏面覺得好驚。太堅強啊,衝鋒毋得啊,一千個導彈炸毋冧佢。又恁下佢寫本日記係畀邊個。畀未來,畀過去。畀個可能虛僞己時代。喺佢前面毋係死刑而係消滅。令日記變得尐灰,自己變蒸氣蒸汽。惟有思警會睇佢寫尐物也,就從存在同記憶己境界抹走著。究竟點樣向未來作請求如果一個無名張紙都無得留?

電幕響十四點。十分鐘佢要走,要十四點半返返工。

好奇,報時己鳴響令佢有多尐信心。佢係得個孤伶伶己鬼,䛅無人會聽到己實話。但係毋知點樣解釋,佢繼續䛅己話,個連續性毋會斷開。毋係令人地聽你而係保持自己己明智先至可以傳住人性己遺產。返返張檯,點點枝筆,寫:
	未來定過去,思想自由、人人有區別同埋毋係單仃感住己時代,眞實、覆水難收己時代:
	由一式己時代,孤獨己時代,大哥己時代,雙恁己時代:你好!

佢恁起,佢係經已死著己。好似係而今,有制定佢尐思考己時候,佢先至有行決心己第一步。佢寫:
	
	思罪毋係引起死亡:思罪卽係死亡。

認著自己係死著己,就知要盡可能長繼續生活。右手兩隻手指漬著尐墨水,就係感己小事可能會賣出你,一個多事己狂熱者(應該都係女人,譬如箇位細粒沙毛個,又或者小說部門己深色頭髮箇個)可能會喺道恁下點解佢食晏飯己時段喺道寫也、點解用枝老爺筆、究竟寫緊尐物也,然之後同某個人暗示。佢去廁所,小心尐擦𠞉尐墨,用著箇個啡色、粗到似砂紙感刮皮匛番鹼,完全適合。

將本日記擠埋喺檯桶道。梗係毋係以爲可以屛得埋佢,而起碼可以知如果有無人發現過。將條頭髮俾兩個封面夾住太擺明。用隻手指尖,𢳂個認得己粒白塵,放著佢喺封面己角上邊,肯定一震就會移位。

	\part{︵二︶}
	\part{︵三︶}
	\end{landscape}	
	\printpagenotes
	\printbibliography
\end{document}
