\documentclass{article}

\usepackage{graphicx} % for \rotatebox

\usepackage{xeCJK}
\newfontlanguage{Chinese}{CHN}
\setCJKmainfont{Sun-ExtA}
\setCJKfamilyfont{songvert}[Script=CJK,Language=Chinese,Vertical=RotatedGlyphs]{Sun-ExtA}

\newcommand*\CJKmovesymbol[1]{\raise.35em\hbox{#1}}
\newcommand*\CJKmove{\punctstyle{plain}% do not modify the spacing between punctuations
  \let\CJKsymbol\CJKmovesymbol
  \let\CJKpunctsymbol\CJKsymbol}

\begin{document}

屈原《离骚》曰:
\begin{center}
\rotatebox{-90}{\fbox{\begin{minipage}{10em}
\CJKfamily{songvert}\CJKmove
『朝发轫于苍梧兮,\\
夕余至乎县圃。\\
欲少留此灵琐兮,\\
日忽忽其将暮。\\
吾令羲和弭节兮,\\
望崦嵫而勿迫。』
\end{minipage}}}
\end{center}
这里羲和便等于一名马车夫,因为
他是御日的,诗人生怕太阳赶快落了,就叫羲和慢一点走。不过话经我
一翻译,显得淘气一点,原文只是一个高贵的身分,另外不表现着什么
个性了。

\end{document}
